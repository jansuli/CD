\documentclass[main.tex]{subfiles}

\begin{document}

\setcounter{chapter}{2}
\setcounter{satz}{21}

\begin{enumerate}[label=(\alph*)]
\setcounter{enumi}{2}
\item Es sei 
$$A = λ \begin{pmatrix}
\cos θ & - \sin θ \\ \sin θ & \cos θ
\end{pmatrix}, \quad λ \in (0,1), θ\in [0,2π).$$
Dann ist $A$ die Hintereinanderausführung einer Rotation um den Winkel $θ$ und einer Rotation um den Faktor $λ$.
Wir erhalten einen \bemph{Fokus}:
\begin{center}


\tikzset{every picture/.style={line width=0.75pt}} %set default line width to 0.75pt        

\begin{tikzpicture}[x=0.75pt,y=0.75pt,yscale=-1,xscale=1]
%uncomment if require: \path (0,300); %set diagram left start at 0, and has height of 300

%Straight Lines [id:da21159494546157775] 
\draw    (320.79,296.58) -- (320.79,30.57) ;
\draw [shift={(320.79,28.57)}, rotate = 450] [color={rgb, 255:red, 0; green, 0; blue, 0 }  ][line width=0.75]    (10.93,-3.29) .. controls (6.95,-1.4) and (3.31,-0.3) .. (0,0) .. controls (3.31,0.3) and (6.95,1.4) .. (10.93,3.29)   ;

%Straight Lines [id:da0805575285181721] 
\draw    (198.68,145.57) -- (457.02,145.57) ;
\draw [shift={(459.02,145.57)}, rotate = 180] [color={rgb, 255:red, 0; green, 0; blue, 0 }  ][line width=0.75]    (10.93,-3.29) .. controls (6.95,-1.4) and (3.31,-0.3) .. (0,0) .. controls (3.31,0.3) and (6.95,1.4) .. (10.93,3.29)   ;

%Curve Lines [id:da7603871315290556] 
\draw    (401.79,60.57) .. controls (318.21,36.69) and (227.7,96.96) .. (244.52,205.93) ;
\draw [shift={(244.79,207.57)}, rotate = 260.71] [color={rgb, 255:red, 0; green, 0; blue, 0 }  ][line width=0.75]    (10.93,-3.29) .. controls (6.95,-1.4) and (3.31,-0.3) .. (0,0) .. controls (3.31,0.3) and (6.95,1.4) .. (10.93,3.29)   ;

%Curve Lines [id:da1558828305787563] 
\draw    (244.79,207.57) .. controls (282.6,300.11) and (396.64,262.95) .. (386.94,186.72) ;
\draw [shift={(386.79,185.57)}, rotate = 441.87] [color={rgb, 255:red, 0; green, 0; blue, 0 }  ][line width=0.75]    (10.93,-3.29) .. controls (6.95,-1.4) and (3.31,-0.3) .. (0,0) .. controls (3.31,0.3) and (6.95,1.4) .. (10.93,3.29)   ;

%Curve Lines [id:da499841546751693] 
\draw    (386.79,185.57) .. controls (384.83,128.73) and (350.21,112.23) .. (323.42,116.3) ;
\draw [shift={(321.79,116.57)}, rotate = 349.51] [color={rgb, 255:red, 0; green, 0; blue, 0 }  ][line width=0.75]    (10.93,-3.29) .. controls (6.95,-1.4) and (3.31,-0.3) .. (0,0) .. controls (3.31,0.3) and (6.95,1.4) .. (10.93,3.29)   ;

%Curve Lines [id:da37382055669303793] 
\draw    (321.79,116.57) .. controls (292.24,123.47) and (300.52,176.93) .. (334.23,160.38) ;
\draw [shift={(335.79,159.57)}, rotate = 511.5] [color={rgb, 255:red, 0; green, 0; blue, 0 }  ][line width=0.75]    (10.93,-3.29) .. controls (6.95,-1.4) and (3.31,-0.3) .. (0,0) .. controls (3.31,0.3) and (6.95,1.4) .. (10.93,3.29)   ;

\draw  [color={rgb, 255:red, 255; green, 0; blue, 0 }  ,draw opacity=1 ] (244.44,212.38) -- (253.71,221.65)(253,213.09) -- (245.15,220.94) ;
\draw  [color={rgb, 255:red, 255; green, 0; blue, 0 }  ,draw opacity=1 ] (356.44,50.38) -- (365.71,59.65)(365,51.09) -- (357.15,58.94) ;
\draw  [color={rgb, 255:red, 255; green, 0; blue, 0 }  ,draw opacity=1 ] (381.44,205.38) -- (390.71,214.65)(390,206.09) -- (382.15,213.94) ;
\draw  [color={rgb, 255:red, 255; green, 0; blue, 0 }  ,draw opacity=1 ] (342.44,114.38) -- (351.71,123.65)(351,115.09) -- (343.15,122.94) ;

% Text Node
\draw (469,148) node   {$x$};
% Text Node
\draw (334,27) node   {$y$};


\end{tikzpicture}

\end{center}
\end{enumerate}

\begin{mydef}\label{2.22}
Eine lineare Abbildung ist \bemph{hyperbolisch}, falls es keinen Eigenwert mit Betrag 1 gibt. 
\end{mydef}
Zu einem reellen Eigenwert $λ$ von $A$ bezeichne $E_λ$ den zugehörigen (verallgemeinerten) Eigenraum.\\
Zu einem konjugiert komplexen Eigenwertpaar $λ,\ov λ$ bezeichne $E_{λ,\ov λ}$ den Real- und Imaginärteil von Eigenvektoren aufgespannten Unterraum. Wir setzen damit
\begin{align*}
E^{-} &= E^{-} (A) = \bigoplus_{|λ|<1} E_λ \oplus \bigoplus_{|λ|<1} E_{λ,\ov λ},\\
E^{+} &= E^{+} (A) = \bigoplus_{|λ|>1} E_λ \oplus \bigoplus_{|λ|>1} E_{λ,\ov λ},\\
E^0 &= E^0(A) = \bigoplus_{|λ|=1} E_λ \oplus \bigoplus_{|λ|=1} E_{λ,\ov λ}.\\
\end{align*}

\begin{bem*}$ $\\[-1em]
\begin{enumerate}
[label=(\alph*)]
% a fehlt
\item Falls $A$ invertierbar ist, so ist $E^+(A) = E^-(A^{-1})$.
\item Die Räume $E^-, E^+, E^0$ sind invariant unter $A$ und es gilt 
$$ℝ^n = E^- \oplus E^+ \oplus E^0.$$
\item $A$ ist \emph{hyperbolisch} genau dann, wenn $E^0 = \{ 0\}$. Das gilt genau dann, wenn $ℝ^n = E^- \oplus E^+.$
\end{enumerate}
\end{bem*}

Mit Satz \ref{2.18} folgt
\begin{korollar}\label{2.23}
Es existiert eine Norm, so dass die Restriktion von $A$ auf $E^-(A)$ eine kontrahierende Abbildung ist.\\
Ist $A$ invertierbar, so ist die Restriktion von $A^{-1}$ auf $E^+(A)$ kontrahierend.
\end{korollar}

\begin{mydef}\label{2.24}
$E^-(A)$ ist der \bemph{stabile Unterraum}, $E^+(A)$ ist der \bemph{instabile Unterraum}.
\end{mydef}

\begin{satz}\label{2.25}
$A$ sei hyperbolisch. Dann gilt:
\begin{enumerate}
[label=(\alph*)]
\item \label{2.25.1} Für jedes $v \in E^-$ konvergiert $A^n v$ exponentiell gegen den Ursprung für $n\to ∞$. Ist $A$ invertierbar, so geht $A^n v$ exponentiell gegen $∞$ für $n\to -∞$ und $v\ne 0$.
\item \label{2.25.2} Für jedes $v\in E^+$ konvergiert $A^n v$ exponentiell gegen Unendlich ($v\ne 0$) für $n\to ∞$. Ist $A$ invertierbar, dann konvergiert $A^n v$ exponentiell gegen den Ursprung für $n\to -∞$.
\item \label{2.25.3} Für jedes $v\in ℝ^n\setminus ( E^+ \cup E^-)$ konvergiert $A^n v$ exponentiell gegen Unendlich für $n\to ∞$ und auch $n\to -∞$, falls $A$ invertierbar ist.
\end{enumerate}
\end{satz}

\begin{proof}
(a) und (b) sind beide klar.\\
Sei also $v\in ℝ^n \setminus (E^- \cup E^+)$. Da $A$ hyperbolisch ist, gilt
$$v = v^- + v^+ \text{ mit } \ov v \in E^-, v^+ \in E^+, v^-, v^+ \ne 0.$$
Wir erhalten für große $n$:
\begin{align}
\| A^n v \| &= \| A^n ( v^- + v^+ ) \| \nonumber \\
& \ge \| A^n v^+ \| - \|A^n v^- \|\nonumber \\
& \ge λ^n c \| v^+ \| - \tilde{λ}^n c ' \| v^- \| \label{eqn:2.25.1} \tag{$\star$}\\
& \ge λ^n c'',\nonumber
\end{align}
wobei $\tilde λ < 1 , λ > 1, c,c', c'' > 0$. 
Für \eqref{eqn:2.25.1} benutze \ref{2.9}:
$$\| A^n v^- \| \le c' \tilde λ^n \| v^- \|$$
sowie
$$\|A^{-n}v^+\| \le λ ν^n \| v^+ \|.$$
Daraus folgt
\begin{align*}
\|v^+ \| &= \| A^{-n} A^n v^+ \| \le λν^n \| A v^+ \| \stackrel{ν<1}\Rightarrow \\
\|A^n v^+ \| &\ge \frac{1}{ν^n} \frac{1}{λ} \| v^+ \| = : cλ^n \| v^+ \|.
\end{align*}
$c''$ kann man z.B. als 
$$c'' = c\frac{ \| v^+  \|}{2}.$$
\end{proof}

\begin{bsp}
\label{2.26}
Es seien $0<μ<1<λ$ und $A = \begin{pmatrix}
λ&0\\ 0& μ
\end{pmatrix}$. Dann gilt $\dim E^- = \dim E^+ = 1$ und man erhält:\\
Einen \bemph{Sattelpunkt}. Gilt speziell $λμ = 1$, so sind die invarianten Kurven die Hyperbeln $xy = \mathrm{const}$.
Daher spricht man von „hyperbolisch“.
\begin{center}
%% Creator: Matplotlib, PGF backend
%%
%% To include the figure in your LaTeX document, write
%%   \input{<filename>.pgf}
%%
%% Make sure the required packages are loaded in your preamble
%%   \usepackage{pgf}
%%
%% Figures using additional raster images can only be included by \input if
%% they are in the same directory as the main LaTeX file. For loading figures
%% from other directories you can use the `import` package
%%   \usepackage{import}
%% and then include the figures with
%%   \import{<path to file>}{<filename>.pgf}
%%
%% Matplotlib used the following preamble
%%   \usepackage{fontspec}
%%   \setmainfont{DejaVu Serif}
%%   \setsansfont{DejaVu Sans}
%%   \setmonofont{DejaVu Sans Mono}
%%
\begingroup%
\makeatletter%
\begin{pgfpicture}%
\pgfpathrectangle{\pgfpointorigin}{\pgfqpoint{6.000000in}{3.000000in}}%
\pgfusepath{use as bounding box, clip}%
\begin{pgfscope}%
\pgfsetbuttcap%
\pgfsetmiterjoin%
\definecolor{currentfill}{rgb}{1.000000,1.000000,1.000000}%
\pgfsetfillcolor{currentfill}%
\pgfsetlinewidth{0.000000pt}%
\definecolor{currentstroke}{rgb}{1.000000,1.000000,1.000000}%
\pgfsetstrokecolor{currentstroke}%
\pgfsetdash{}{0pt}%
\pgfpathmoveto{\pgfqpoint{0.000000in}{0.000000in}}%
\pgfpathlineto{\pgfqpoint{6.000000in}{0.000000in}}%
\pgfpathlineto{\pgfqpoint{6.000000in}{3.000000in}}%
\pgfpathlineto{\pgfqpoint{0.000000in}{3.000000in}}%
\pgfpathclose%
\pgfusepath{fill}%
\end{pgfscope}%
\begin{pgfscope}%
\pgfsetbuttcap%
\pgfsetmiterjoin%
\definecolor{currentfill}{rgb}{1.000000,1.000000,1.000000}%
\pgfsetfillcolor{currentfill}%
\pgfsetlinewidth{0.000000pt}%
\definecolor{currentstroke}{rgb}{0.000000,0.000000,0.000000}%
\pgfsetstrokecolor{currentstroke}%
\pgfsetstrokeopacity{0.000000}%
\pgfsetdash{}{0pt}%
\pgfpathmoveto{\pgfqpoint{0.750000in}{0.375000in}}%
\pgfpathlineto{\pgfqpoint{5.400000in}{0.375000in}}%
\pgfpathlineto{\pgfqpoint{5.400000in}{2.640000in}}%
\pgfpathlineto{\pgfqpoint{0.750000in}{2.640000in}}%
\pgfpathclose%
\pgfusepath{fill}%
\end{pgfscope}%
\begin{pgfscope}%
\pgfpathrectangle{\pgfqpoint{0.750000in}{0.375000in}}{\pgfqpoint{4.650000in}{2.265000in}}%
\pgfusepath{clip}%
\pgfsetbuttcap%
\pgfsetroundjoin%
\definecolor{currentfill}{rgb}{0.121569,0.466667,0.705882}%
\pgfsetfillcolor{currentfill}%
\pgfsetlinewidth{1.003750pt}%
\definecolor{currentstroke}{rgb}{0.121569,0.466667,0.705882}%
\pgfsetstrokecolor{currentstroke}%
\pgfsetdash{}{0pt}%
\pgfpathmoveto{\pgfqpoint{3.239962in}{2.454418in}}%
\pgfpathcurveto{\pgfqpoint{3.251012in}{2.454418in}}{\pgfqpoint{3.261611in}{2.458809in}}{\pgfqpoint{3.269424in}{2.466622in}}%
\pgfpathcurveto{\pgfqpoint{3.277238in}{2.474436in}}{\pgfqpoint{3.281628in}{2.485035in}}{\pgfqpoint{3.281628in}{2.496085in}}%
\pgfpathcurveto{\pgfqpoint{3.281628in}{2.507135in}}{\pgfqpoint{3.277238in}{2.517734in}}{\pgfqpoint{3.269424in}{2.525548in}}%
\pgfpathcurveto{\pgfqpoint{3.261611in}{2.533361in}}{\pgfqpoint{3.251012in}{2.537752in}}{\pgfqpoint{3.239962in}{2.537752in}}%
\pgfpathcurveto{\pgfqpoint{3.228911in}{2.537752in}}{\pgfqpoint{3.218312in}{2.533361in}}{\pgfqpoint{3.210499in}{2.525548in}}%
\pgfpathcurveto{\pgfqpoint{3.202685in}{2.517734in}}{\pgfqpoint{3.198295in}{2.507135in}}{\pgfqpoint{3.198295in}{2.496085in}}%
\pgfpathcurveto{\pgfqpoint{3.198295in}{2.485035in}}{\pgfqpoint{3.202685in}{2.474436in}}{\pgfqpoint{3.210499in}{2.466622in}}%
\pgfpathcurveto{\pgfqpoint{3.218312in}{2.458809in}}{\pgfqpoint{3.228911in}{2.454418in}}{\pgfqpoint{3.239962in}{2.454418in}}%
\pgfpathclose%
\pgfusepath{stroke,fill}%
\end{pgfscope}%
\begin{pgfscope}%
\pgfpathrectangle{\pgfqpoint{0.750000in}{0.375000in}}{\pgfqpoint{4.650000in}{2.265000in}}%
\pgfusepath{clip}%
\pgfsetbuttcap%
\pgfsetroundjoin%
\definecolor{currentfill}{rgb}{0.121569,0.466667,0.705882}%
\pgfsetfillcolor{currentfill}%
\pgfsetlinewidth{1.003750pt}%
\definecolor{currentstroke}{rgb}{0.121569,0.466667,0.705882}%
\pgfsetstrokecolor{currentstroke}%
\pgfsetdash{}{0pt}%
\pgfpathmoveto{\pgfqpoint{3.297641in}{1.960023in}}%
\pgfpathcurveto{\pgfqpoint{3.308691in}{1.960023in}}{\pgfqpoint{3.319290in}{1.964413in}}{\pgfqpoint{3.327104in}{1.972227in}}%
\pgfpathcurveto{\pgfqpoint{3.334917in}{1.980040in}}{\pgfqpoint{3.339308in}{1.990639in}}{\pgfqpoint{3.339308in}{2.001689in}}%
\pgfpathcurveto{\pgfqpoint{3.339308in}{2.012740in}}{\pgfqpoint{3.334917in}{2.023339in}}{\pgfqpoint{3.327104in}{2.031152in}}%
\pgfpathcurveto{\pgfqpoint{3.319290in}{2.038966in}}{\pgfqpoint{3.308691in}{2.043356in}}{\pgfqpoint{3.297641in}{2.043356in}}%
\pgfpathcurveto{\pgfqpoint{3.286591in}{2.043356in}}{\pgfqpoint{3.275992in}{2.038966in}}{\pgfqpoint{3.268178in}{2.031152in}}%
\pgfpathcurveto{\pgfqpoint{3.260365in}{2.023339in}}{\pgfqpoint{3.255974in}{2.012740in}}{\pgfqpoint{3.255974in}{2.001689in}}%
\pgfpathcurveto{\pgfqpoint{3.255974in}{1.990639in}}{\pgfqpoint{3.260365in}{1.980040in}}{\pgfqpoint{3.268178in}{1.972227in}}%
\pgfpathcurveto{\pgfqpoint{3.275992in}{1.964413in}}{\pgfqpoint{3.286591in}{1.960023in}}{\pgfqpoint{3.297641in}{1.960023in}}%
\pgfpathclose%
\pgfusepath{stroke,fill}%
\end{pgfscope}%
\begin{pgfscope}%
\pgfpathrectangle{\pgfqpoint{0.750000in}{0.375000in}}{\pgfqpoint{4.650000in}{2.265000in}}%
\pgfusepath{clip}%
\pgfsetbuttcap%
\pgfsetroundjoin%
\definecolor{currentfill}{rgb}{0.121569,0.466667,0.705882}%
\pgfsetfillcolor{currentfill}%
\pgfsetlinewidth{1.003750pt}%
\definecolor{currentstroke}{rgb}{0.121569,0.466667,0.705882}%
\pgfsetstrokecolor{currentstroke}%
\pgfsetdash{}{0pt}%
\pgfpathmoveto{\pgfqpoint{3.375508in}{1.712825in}}%
\pgfpathcurveto{\pgfqpoint{3.386558in}{1.712825in}}{\pgfqpoint{3.397157in}{1.717215in}}{\pgfqpoint{3.404971in}{1.725029in}}%
\pgfpathcurveto{\pgfqpoint{3.412785in}{1.732842in}}{\pgfqpoint{3.417175in}{1.743442in}}{\pgfqpoint{3.417175in}{1.754492in}}%
\pgfpathcurveto{\pgfqpoint{3.417175in}{1.765542in}}{\pgfqpoint{3.412785in}{1.776141in}}{\pgfqpoint{3.404971in}{1.783954in}}%
\pgfpathcurveto{\pgfqpoint{3.397157in}{1.791768in}}{\pgfqpoint{3.386558in}{1.796158in}}{\pgfqpoint{3.375508in}{1.796158in}}%
\pgfpathcurveto{\pgfqpoint{3.364458in}{1.796158in}}{\pgfqpoint{3.353859in}{1.791768in}}{\pgfqpoint{3.346046in}{1.783954in}}%
\pgfpathcurveto{\pgfqpoint{3.338232in}{1.776141in}}{\pgfqpoint{3.333842in}{1.765542in}}{\pgfqpoint{3.333842in}{1.754492in}}%
\pgfpathcurveto{\pgfqpoint{3.333842in}{1.743442in}}{\pgfqpoint{3.338232in}{1.732842in}}{\pgfqpoint{3.346046in}{1.725029in}}%
\pgfpathcurveto{\pgfqpoint{3.353859in}{1.717215in}}{\pgfqpoint{3.364458in}{1.712825in}}{\pgfqpoint{3.375508in}{1.712825in}}%
\pgfpathclose%
\pgfusepath{stroke,fill}%
\end{pgfscope}%
\begin{pgfscope}%
\pgfpathrectangle{\pgfqpoint{0.750000in}{0.375000in}}{\pgfqpoint{4.650000in}{2.265000in}}%
\pgfusepath{clip}%
\pgfsetbuttcap%
\pgfsetroundjoin%
\definecolor{currentfill}{rgb}{0.121569,0.466667,0.705882}%
\pgfsetfillcolor{currentfill}%
\pgfsetlinewidth{1.003750pt}%
\definecolor{currentstroke}{rgb}{0.121569,0.466667,0.705882}%
\pgfsetstrokecolor{currentstroke}%
\pgfsetdash{}{0pt}%
\pgfpathmoveto{\pgfqpoint{3.480629in}{1.589226in}}%
\pgfpathcurveto{\pgfqpoint{3.491679in}{1.589226in}}{\pgfqpoint{3.502278in}{1.593616in}}{\pgfqpoint{3.510092in}{1.601430in}}%
\pgfpathcurveto{\pgfqpoint{3.517906in}{1.609244in}}{\pgfqpoint{3.522296in}{1.619843in}}{\pgfqpoint{3.522296in}{1.630893in}}%
\pgfpathcurveto{\pgfqpoint{3.522296in}{1.641943in}}{\pgfqpoint{3.517906in}{1.652542in}}{\pgfqpoint{3.510092in}{1.660356in}}%
\pgfpathcurveto{\pgfqpoint{3.502278in}{1.668169in}}{\pgfqpoint{3.491679in}{1.672559in}}{\pgfqpoint{3.480629in}{1.672559in}}%
\pgfpathcurveto{\pgfqpoint{3.469579in}{1.672559in}}{\pgfqpoint{3.458980in}{1.668169in}}{\pgfqpoint{3.451166in}{1.660356in}}%
\pgfpathcurveto{\pgfqpoint{3.443353in}{1.652542in}}{\pgfqpoint{3.438962in}{1.641943in}}{\pgfqpoint{3.438962in}{1.630893in}}%
\pgfpathcurveto{\pgfqpoint{3.438962in}{1.619843in}}{\pgfqpoint{3.443353in}{1.609244in}}{\pgfqpoint{3.451166in}{1.601430in}}%
\pgfpathcurveto{\pgfqpoint{3.458980in}{1.593616in}}{\pgfqpoint{3.469579in}{1.589226in}}{\pgfqpoint{3.480629in}{1.589226in}}%
\pgfpathclose%
\pgfusepath{stroke,fill}%
\end{pgfscope}%
\begin{pgfscope}%
\pgfpathrectangle{\pgfqpoint{0.750000in}{0.375000in}}{\pgfqpoint{4.650000in}{2.265000in}}%
\pgfusepath{clip}%
\pgfsetbuttcap%
\pgfsetroundjoin%
\definecolor{currentfill}{rgb}{0.121569,0.466667,0.705882}%
\pgfsetfillcolor{currentfill}%
\pgfsetlinewidth{1.003750pt}%
\definecolor{currentstroke}{rgb}{0.121569,0.466667,0.705882}%
\pgfsetstrokecolor{currentstroke}%
\pgfsetdash{}{0pt}%
\pgfpathmoveto{\pgfqpoint{3.622542in}{1.527427in}}%
\pgfpathcurveto{\pgfqpoint{3.633592in}{1.527427in}}{\pgfqpoint{3.644191in}{1.531817in}}{\pgfqpoint{3.652005in}{1.539631in}}%
\pgfpathcurveto{\pgfqpoint{3.659819in}{1.547444in}}{\pgfqpoint{3.664209in}{1.558043in}}{\pgfqpoint{3.664209in}{1.569093in}}%
\pgfpathcurveto{\pgfqpoint{3.664209in}{1.580143in}}{\pgfqpoint{3.659819in}{1.590742in}}{\pgfqpoint{3.652005in}{1.598556in}}%
\pgfpathcurveto{\pgfqpoint{3.644191in}{1.606370in}}{\pgfqpoint{3.633592in}{1.610760in}}{\pgfqpoint{3.622542in}{1.610760in}}%
\pgfpathcurveto{\pgfqpoint{3.611492in}{1.610760in}}{\pgfqpoint{3.600893in}{1.606370in}}{\pgfqpoint{3.593080in}{1.598556in}}%
\pgfpathcurveto{\pgfqpoint{3.585266in}{1.590742in}}{\pgfqpoint{3.580876in}{1.580143in}}{\pgfqpoint{3.580876in}{1.569093in}}%
\pgfpathcurveto{\pgfqpoint{3.580876in}{1.558043in}}{\pgfqpoint{3.585266in}{1.547444in}}{\pgfqpoint{3.593080in}{1.539631in}}%
\pgfpathcurveto{\pgfqpoint{3.600893in}{1.531817in}}{\pgfqpoint{3.611492in}{1.527427in}}{\pgfqpoint{3.622542in}{1.527427in}}%
\pgfpathclose%
\pgfusepath{stroke,fill}%
\end{pgfscope}%
\begin{pgfscope}%
\pgfpathrectangle{\pgfqpoint{0.750000in}{0.375000in}}{\pgfqpoint{4.650000in}{2.265000in}}%
\pgfusepath{clip}%
\pgfsetbuttcap%
\pgfsetroundjoin%
\definecolor{currentfill}{rgb}{0.121569,0.466667,0.705882}%
\pgfsetfillcolor{currentfill}%
\pgfsetlinewidth{1.003750pt}%
\definecolor{currentstroke}{rgb}{0.121569,0.466667,0.705882}%
\pgfsetstrokecolor{currentstroke}%
\pgfsetdash{}{0pt}%
\pgfpathmoveto{\pgfqpoint{3.814125in}{1.496527in}}%
\pgfpathcurveto{\pgfqpoint{3.825175in}{1.496527in}}{\pgfqpoint{3.835774in}{1.500917in}}{\pgfqpoint{3.843588in}{1.508731in}}%
\pgfpathcurveto{\pgfqpoint{3.851401in}{1.516544in}}{\pgfqpoint{3.855792in}{1.527143in}}{\pgfqpoint{3.855792in}{1.538194in}}%
\pgfpathcurveto{\pgfqpoint{3.855792in}{1.549244in}}{\pgfqpoint{3.851401in}{1.559843in}}{\pgfqpoint{3.843588in}{1.567656in}}%
\pgfpathcurveto{\pgfqpoint{3.835774in}{1.575470in}}{\pgfqpoint{3.825175in}{1.579860in}}{\pgfqpoint{3.814125in}{1.579860in}}%
\pgfpathcurveto{\pgfqpoint{3.803075in}{1.579860in}}{\pgfqpoint{3.792476in}{1.575470in}}{\pgfqpoint{3.784662in}{1.567656in}}%
\pgfpathcurveto{\pgfqpoint{3.776849in}{1.559843in}}{\pgfqpoint{3.772458in}{1.549244in}}{\pgfqpoint{3.772458in}{1.538194in}}%
\pgfpathcurveto{\pgfqpoint{3.772458in}{1.527143in}}{\pgfqpoint{3.776849in}{1.516544in}}{\pgfqpoint{3.784662in}{1.508731in}}%
\pgfpathcurveto{\pgfqpoint{3.792476in}{1.500917in}}{\pgfqpoint{3.803075in}{1.496527in}}{\pgfqpoint{3.814125in}{1.496527in}}%
\pgfpathclose%
\pgfusepath{stroke,fill}%
\end{pgfscope}%
\begin{pgfscope}%
\pgfpathrectangle{\pgfqpoint{0.750000in}{0.375000in}}{\pgfqpoint{4.650000in}{2.265000in}}%
\pgfusepath{clip}%
\pgfsetbuttcap%
\pgfsetroundjoin%
\definecolor{currentfill}{rgb}{0.121569,0.466667,0.705882}%
\pgfsetfillcolor{currentfill}%
\pgfsetlinewidth{1.003750pt}%
\definecolor{currentstroke}{rgb}{0.121569,0.466667,0.705882}%
\pgfsetstrokecolor{currentstroke}%
\pgfsetdash{}{0pt}%
\pgfpathmoveto{\pgfqpoint{4.072762in}{1.481077in}}%
\pgfpathcurveto{\pgfqpoint{4.083812in}{1.481077in}}{\pgfqpoint{4.094411in}{1.485467in}}{\pgfqpoint{4.102225in}{1.493281in}}%
\pgfpathcurveto{\pgfqpoint{4.110038in}{1.501095in}}{\pgfqpoint{4.114428in}{1.511694in}}{\pgfqpoint{4.114428in}{1.522744in}}%
\pgfpathcurveto{\pgfqpoint{4.114428in}{1.533794in}}{\pgfqpoint{4.110038in}{1.544393in}}{\pgfqpoint{4.102225in}{1.552207in}}%
\pgfpathcurveto{\pgfqpoint{4.094411in}{1.560020in}}{\pgfqpoint{4.083812in}{1.564410in}}{\pgfqpoint{4.072762in}{1.564410in}}%
\pgfpathcurveto{\pgfqpoint{4.061712in}{1.564410in}}{\pgfqpoint{4.051113in}{1.560020in}}{\pgfqpoint{4.043299in}{1.552207in}}%
\pgfpathcurveto{\pgfqpoint{4.035485in}{1.544393in}}{\pgfqpoint{4.031095in}{1.533794in}}{\pgfqpoint{4.031095in}{1.522744in}}%
\pgfpathcurveto{\pgfqpoint{4.031095in}{1.511694in}}{\pgfqpoint{4.035485in}{1.501095in}}{\pgfqpoint{4.043299in}{1.493281in}}%
\pgfpathcurveto{\pgfqpoint{4.051113in}{1.485467in}}{\pgfqpoint{4.061712in}{1.481077in}}{\pgfqpoint{4.072762in}{1.481077in}}%
\pgfpathclose%
\pgfusepath{stroke,fill}%
\end{pgfscope}%
\begin{pgfscope}%
\pgfpathrectangle{\pgfqpoint{0.750000in}{0.375000in}}{\pgfqpoint{4.650000in}{2.265000in}}%
\pgfusepath{clip}%
\pgfsetbuttcap%
\pgfsetroundjoin%
\definecolor{currentfill}{rgb}{0.121569,0.466667,0.705882}%
\pgfsetfillcolor{currentfill}%
\pgfsetlinewidth{1.003750pt}%
\definecolor{currentstroke}{rgb}{0.121569,0.466667,0.705882}%
\pgfsetstrokecolor{currentstroke}%
\pgfsetdash{}{0pt}%
\pgfpathmoveto{\pgfqpoint{4.421921in}{1.473352in}}%
\pgfpathcurveto{\pgfqpoint{4.432971in}{1.473352in}}{\pgfqpoint{4.443570in}{1.477742in}}{\pgfqpoint{4.451384in}{1.485556in}}%
\pgfpathcurveto{\pgfqpoint{4.459198in}{1.493370in}}{\pgfqpoint{4.463588in}{1.503969in}}{\pgfqpoint{4.463588in}{1.515019in}}%
\pgfpathcurveto{\pgfqpoint{4.463588in}{1.526069in}}{\pgfqpoint{4.459198in}{1.536668in}}{\pgfqpoint{4.451384in}{1.544482in}}%
\pgfpathcurveto{\pgfqpoint{4.443570in}{1.552295in}}{\pgfqpoint{4.432971in}{1.556685in}}{\pgfqpoint{4.421921in}{1.556685in}}%
\pgfpathcurveto{\pgfqpoint{4.410871in}{1.556685in}}{\pgfqpoint{4.400272in}{1.552295in}}{\pgfqpoint{4.392459in}{1.544482in}}%
\pgfpathcurveto{\pgfqpoint{4.384645in}{1.536668in}}{\pgfqpoint{4.380255in}{1.526069in}}{\pgfqpoint{4.380255in}{1.515019in}}%
\pgfpathcurveto{\pgfqpoint{4.380255in}{1.503969in}}{\pgfqpoint{4.384645in}{1.493370in}}{\pgfqpoint{4.392459in}{1.485556in}}%
\pgfpathcurveto{\pgfqpoint{4.400272in}{1.477742in}}{\pgfqpoint{4.410871in}{1.473352in}}{\pgfqpoint{4.421921in}{1.473352in}}%
\pgfpathclose%
\pgfusepath{stroke,fill}%
\end{pgfscope}%
\begin{pgfscope}%
\pgfpathrectangle{\pgfqpoint{0.750000in}{0.375000in}}{\pgfqpoint{4.650000in}{2.265000in}}%
\pgfusepath{clip}%
\pgfsetbuttcap%
\pgfsetroundjoin%
\definecolor{currentfill}{rgb}{0.121569,0.466667,0.705882}%
\pgfsetfillcolor{currentfill}%
\pgfsetlinewidth{1.003750pt}%
\definecolor{currentstroke}{rgb}{0.121569,0.466667,0.705882}%
\pgfsetstrokecolor{currentstroke}%
\pgfsetdash{}{0pt}%
\pgfpathmoveto{\pgfqpoint{4.893287in}{1.469490in}}%
\pgfpathcurveto{\pgfqpoint{4.904337in}{1.469490in}}{\pgfqpoint{4.914936in}{1.473880in}}{\pgfqpoint{4.922750in}{1.481694in}}%
\pgfpathcurveto{\pgfqpoint{4.930563in}{1.489507in}}{\pgfqpoint{4.934953in}{1.500106in}}{\pgfqpoint{4.934953in}{1.511156in}}%
\pgfpathcurveto{\pgfqpoint{4.934953in}{1.522206in}}{\pgfqpoint{4.930563in}{1.532806in}}{\pgfqpoint{4.922750in}{1.540619in}}%
\pgfpathcurveto{\pgfqpoint{4.914936in}{1.548433in}}{\pgfqpoint{4.904337in}{1.552823in}}{\pgfqpoint{4.893287in}{1.552823in}}%
\pgfpathcurveto{\pgfqpoint{4.882237in}{1.552823in}}{\pgfqpoint{4.871638in}{1.548433in}}{\pgfqpoint{4.863824in}{1.540619in}}%
\pgfpathcurveto{\pgfqpoint{4.856010in}{1.532806in}}{\pgfqpoint{4.851620in}{1.522206in}}{\pgfqpoint{4.851620in}{1.511156in}}%
\pgfpathcurveto{\pgfqpoint{4.851620in}{1.500106in}}{\pgfqpoint{4.856010in}{1.489507in}}{\pgfqpoint{4.863824in}{1.481694in}}%
\pgfpathcurveto{\pgfqpoint{4.871638in}{1.473880in}}{\pgfqpoint{4.882237in}{1.469490in}}{\pgfqpoint{4.893287in}{1.469490in}}%
\pgfpathclose%
\pgfusepath{stroke,fill}%
\end{pgfscope}%
\begin{pgfscope}%
\pgfpathrectangle{\pgfqpoint{0.750000in}{0.375000in}}{\pgfqpoint{4.650000in}{2.265000in}}%
\pgfusepath{clip}%
\pgfsetbuttcap%
\pgfsetroundjoin%
\definecolor{currentfill}{rgb}{1.000000,0.498039,0.054902}%
\pgfsetfillcolor{currentfill}%
\pgfsetlinewidth{1.003750pt}%
\definecolor{currentstroke}{rgb}{1.000000,0.498039,0.054902}%
\pgfsetstrokecolor{currentstroke}%
\pgfsetdash{}{0pt}%
\pgfpathmoveto{\pgfqpoint{2.910365in}{0.476836in}}%
\pgfpathcurveto{\pgfqpoint{2.921415in}{0.476836in}}{\pgfqpoint{2.932014in}{0.481226in}}{\pgfqpoint{2.939827in}{0.489040in}}%
\pgfpathcurveto{\pgfqpoint{2.947641in}{0.496854in}}{\pgfqpoint{2.952031in}{0.507453in}}{\pgfqpoint{2.952031in}{0.518503in}}%
\pgfpathcurveto{\pgfqpoint{2.952031in}{0.529553in}}{\pgfqpoint{2.947641in}{0.540152in}}{\pgfqpoint{2.939827in}{0.547966in}}%
\pgfpathcurveto{\pgfqpoint{2.932014in}{0.555779in}}{\pgfqpoint{2.921415in}{0.560169in}}{\pgfqpoint{2.910365in}{0.560169in}}%
\pgfpathcurveto{\pgfqpoint{2.899314in}{0.560169in}}{\pgfqpoint{2.888715in}{0.555779in}}{\pgfqpoint{2.880902in}{0.547966in}}%
\pgfpathcurveto{\pgfqpoint{2.873088in}{0.540152in}}{\pgfqpoint{2.868698in}{0.529553in}}{\pgfqpoint{2.868698in}{0.518503in}}%
\pgfpathcurveto{\pgfqpoint{2.868698in}{0.507453in}}{\pgfqpoint{2.873088in}{0.496854in}}{\pgfqpoint{2.880902in}{0.489040in}}%
\pgfpathcurveto{\pgfqpoint{2.888715in}{0.481226in}}{\pgfqpoint{2.899314in}{0.476836in}}{\pgfqpoint{2.910365in}{0.476836in}}%
\pgfpathclose%
\pgfusepath{stroke,fill}%
\end{pgfscope}%
\begin{pgfscope}%
\pgfpathrectangle{\pgfqpoint{0.750000in}{0.375000in}}{\pgfqpoint{4.650000in}{2.265000in}}%
\pgfusepath{clip}%
\pgfsetbuttcap%
\pgfsetroundjoin%
\definecolor{currentfill}{rgb}{1.000000,0.498039,0.054902}%
\pgfsetfillcolor{currentfill}%
\pgfsetlinewidth{1.003750pt}%
\definecolor{currentstroke}{rgb}{1.000000,0.498039,0.054902}%
\pgfsetstrokecolor{currentstroke}%
\pgfsetdash{}{0pt}%
\pgfpathmoveto{\pgfqpoint{2.852685in}{0.971232in}}%
\pgfpathcurveto{\pgfqpoint{2.863735in}{0.971232in}}{\pgfqpoint{2.874334in}{0.975622in}}{\pgfqpoint{2.882148in}{0.983436in}}%
\pgfpathcurveto{\pgfqpoint{2.889961in}{0.991249in}}{\pgfqpoint{2.894352in}{1.001848in}}{\pgfqpoint{2.894352in}{1.012898in}}%
\pgfpathcurveto{\pgfqpoint{2.894352in}{1.023948in}}{\pgfqpoint{2.889961in}{1.034548in}}{\pgfqpoint{2.882148in}{1.042361in}}%
\pgfpathcurveto{\pgfqpoint{2.874334in}{1.050175in}}{\pgfqpoint{2.863735in}{1.054565in}}{\pgfqpoint{2.852685in}{1.054565in}}%
\pgfpathcurveto{\pgfqpoint{2.841635in}{1.054565in}}{\pgfqpoint{2.831036in}{1.050175in}}{\pgfqpoint{2.823222in}{1.042361in}}%
\pgfpathcurveto{\pgfqpoint{2.815409in}{1.034548in}}{\pgfqpoint{2.811018in}{1.023948in}}{\pgfqpoint{2.811018in}{1.012898in}}%
\pgfpathcurveto{\pgfqpoint{2.811018in}{1.001848in}}{\pgfqpoint{2.815409in}{0.991249in}}{\pgfqpoint{2.823222in}{0.983436in}}%
\pgfpathcurveto{\pgfqpoint{2.831036in}{0.975622in}}{\pgfqpoint{2.841635in}{0.971232in}}{\pgfqpoint{2.852685in}{0.971232in}}%
\pgfpathclose%
\pgfusepath{stroke,fill}%
\end{pgfscope}%
\begin{pgfscope}%
\pgfpathrectangle{\pgfqpoint{0.750000in}{0.375000in}}{\pgfqpoint{4.650000in}{2.265000in}}%
\pgfusepath{clip}%
\pgfsetbuttcap%
\pgfsetroundjoin%
\definecolor{currentfill}{rgb}{1.000000,0.498039,0.054902}%
\pgfsetfillcolor{currentfill}%
\pgfsetlinewidth{1.003750pt}%
\definecolor{currentstroke}{rgb}{1.000000,0.498039,0.054902}%
\pgfsetstrokecolor{currentstroke}%
\pgfsetdash{}{0pt}%
\pgfpathmoveto{\pgfqpoint{2.774818in}{1.218429in}}%
\pgfpathcurveto{\pgfqpoint{2.785868in}{1.218429in}}{\pgfqpoint{2.796467in}{1.222820in}}{\pgfqpoint{2.804281in}{1.230633in}}%
\pgfpathcurveto{\pgfqpoint{2.812094in}{1.238447in}}{\pgfqpoint{2.816484in}{1.249046in}}{\pgfqpoint{2.816484in}{1.260096in}}%
\pgfpathcurveto{\pgfqpoint{2.816484in}{1.271146in}}{\pgfqpoint{2.812094in}{1.281745in}}{\pgfqpoint{2.804281in}{1.289559in}}%
\pgfpathcurveto{\pgfqpoint{2.796467in}{1.297373in}}{\pgfqpoint{2.785868in}{1.301763in}}{\pgfqpoint{2.774818in}{1.301763in}}%
\pgfpathcurveto{\pgfqpoint{2.763768in}{1.301763in}}{\pgfqpoint{2.753169in}{1.297373in}}{\pgfqpoint{2.745355in}{1.289559in}}%
\pgfpathcurveto{\pgfqpoint{2.737541in}{1.281745in}}{\pgfqpoint{2.733151in}{1.271146in}}{\pgfqpoint{2.733151in}{1.260096in}}%
\pgfpathcurveto{\pgfqpoint{2.733151in}{1.249046in}}{\pgfqpoint{2.737541in}{1.238447in}}{\pgfqpoint{2.745355in}{1.230633in}}%
\pgfpathcurveto{\pgfqpoint{2.753169in}{1.222820in}}{\pgfqpoint{2.763768in}{1.218429in}}{\pgfqpoint{2.774818in}{1.218429in}}%
\pgfpathclose%
\pgfusepath{stroke,fill}%
\end{pgfscope}%
\begin{pgfscope}%
\pgfpathrectangle{\pgfqpoint{0.750000in}{0.375000in}}{\pgfqpoint{4.650000in}{2.265000in}}%
\pgfusepath{clip}%
\pgfsetbuttcap%
\pgfsetroundjoin%
\definecolor{currentfill}{rgb}{1.000000,0.498039,0.054902}%
\pgfsetfillcolor{currentfill}%
\pgfsetlinewidth{1.003750pt}%
\definecolor{currentstroke}{rgb}{1.000000,0.498039,0.054902}%
\pgfsetstrokecolor{currentstroke}%
\pgfsetdash{}{0pt}%
\pgfpathmoveto{\pgfqpoint{2.669697in}{1.342028in}}%
\pgfpathcurveto{\pgfqpoint{2.680747in}{1.342028in}}{\pgfqpoint{2.691346in}{1.346419in}}{\pgfqpoint{2.699160in}{1.354232in}}%
\pgfpathcurveto{\pgfqpoint{2.706973in}{1.362046in}}{\pgfqpoint{2.711364in}{1.372645in}}{\pgfqpoint{2.711364in}{1.383695in}}%
\pgfpathcurveto{\pgfqpoint{2.711364in}{1.394745in}}{\pgfqpoint{2.706973in}{1.405344in}}{\pgfqpoint{2.699160in}{1.413158in}}%
\pgfpathcurveto{\pgfqpoint{2.691346in}{1.420971in}}{\pgfqpoint{2.680747in}{1.425362in}}{\pgfqpoint{2.669697in}{1.425362in}}%
\pgfpathcurveto{\pgfqpoint{2.658647in}{1.425362in}}{\pgfqpoint{2.648048in}{1.420971in}}{\pgfqpoint{2.640234in}{1.413158in}}%
\pgfpathcurveto{\pgfqpoint{2.632420in}{1.405344in}}{\pgfqpoint{2.628030in}{1.394745in}}{\pgfqpoint{2.628030in}{1.383695in}}%
\pgfpathcurveto{\pgfqpoint{2.628030in}{1.372645in}}{\pgfqpoint{2.632420in}{1.362046in}}{\pgfqpoint{2.640234in}{1.354232in}}%
\pgfpathcurveto{\pgfqpoint{2.648048in}{1.346419in}}{\pgfqpoint{2.658647in}{1.342028in}}{\pgfqpoint{2.669697in}{1.342028in}}%
\pgfpathclose%
\pgfusepath{stroke,fill}%
\end{pgfscope}%
\begin{pgfscope}%
\pgfpathrectangle{\pgfqpoint{0.750000in}{0.375000in}}{\pgfqpoint{4.650000in}{2.265000in}}%
\pgfusepath{clip}%
\pgfsetbuttcap%
\pgfsetroundjoin%
\definecolor{currentfill}{rgb}{1.000000,0.498039,0.054902}%
\pgfsetfillcolor{currentfill}%
\pgfsetlinewidth{1.003750pt}%
\definecolor{currentstroke}{rgb}{1.000000,0.498039,0.054902}%
\pgfsetstrokecolor{currentstroke}%
\pgfsetdash{}{0pt}%
\pgfpathmoveto{\pgfqpoint{2.527784in}{1.403828in}}%
\pgfpathcurveto{\pgfqpoint{2.538834in}{1.403828in}}{\pgfqpoint{2.549433in}{1.408218in}}{\pgfqpoint{2.557247in}{1.416032in}}%
\pgfpathcurveto{\pgfqpoint{2.565060in}{1.423845in}}{\pgfqpoint{2.569450in}{1.434444in}}{\pgfqpoint{2.569450in}{1.445494in}}%
\pgfpathcurveto{\pgfqpoint{2.569450in}{1.456545in}}{\pgfqpoint{2.565060in}{1.467144in}}{\pgfqpoint{2.557247in}{1.474957in}}%
\pgfpathcurveto{\pgfqpoint{2.549433in}{1.482771in}}{\pgfqpoint{2.538834in}{1.487161in}}{\pgfqpoint{2.527784in}{1.487161in}}%
\pgfpathcurveto{\pgfqpoint{2.516734in}{1.487161in}}{\pgfqpoint{2.506135in}{1.482771in}}{\pgfqpoint{2.498321in}{1.474957in}}%
\pgfpathcurveto{\pgfqpoint{2.490507in}{1.467144in}}{\pgfqpoint{2.486117in}{1.456545in}}{\pgfqpoint{2.486117in}{1.445494in}}%
\pgfpathcurveto{\pgfqpoint{2.486117in}{1.434444in}}{\pgfqpoint{2.490507in}{1.423845in}}{\pgfqpoint{2.498321in}{1.416032in}}%
\pgfpathcurveto{\pgfqpoint{2.506135in}{1.408218in}}{\pgfqpoint{2.516734in}{1.403828in}}{\pgfqpoint{2.527784in}{1.403828in}}%
\pgfpathclose%
\pgfusepath{stroke,fill}%
\end{pgfscope}%
\begin{pgfscope}%
\pgfpathrectangle{\pgfqpoint{0.750000in}{0.375000in}}{\pgfqpoint{4.650000in}{2.265000in}}%
\pgfusepath{clip}%
\pgfsetbuttcap%
\pgfsetroundjoin%
\definecolor{currentfill}{rgb}{1.000000,0.498039,0.054902}%
\pgfsetfillcolor{currentfill}%
\pgfsetlinewidth{1.003750pt}%
\definecolor{currentstroke}{rgb}{1.000000,0.498039,0.054902}%
\pgfsetstrokecolor{currentstroke}%
\pgfsetdash{}{0pt}%
\pgfpathmoveto{\pgfqpoint{2.336201in}{1.434727in}}%
\pgfpathcurveto{\pgfqpoint{2.347251in}{1.434727in}}{\pgfqpoint{2.357850in}{1.439118in}}{\pgfqpoint{2.365664in}{1.446931in}}%
\pgfpathcurveto{\pgfqpoint{2.373477in}{1.454745in}}{\pgfqpoint{2.377868in}{1.465344in}}{\pgfqpoint{2.377868in}{1.476394in}}%
\pgfpathcurveto{\pgfqpoint{2.377868in}{1.487444in}}{\pgfqpoint{2.373477in}{1.498043in}}{\pgfqpoint{2.365664in}{1.505857in}}%
\pgfpathcurveto{\pgfqpoint{2.357850in}{1.513671in}}{\pgfqpoint{2.347251in}{1.518061in}}{\pgfqpoint{2.336201in}{1.518061in}}%
\pgfpathcurveto{\pgfqpoint{2.325151in}{1.518061in}}{\pgfqpoint{2.314552in}{1.513671in}}{\pgfqpoint{2.306738in}{1.505857in}}%
\pgfpathcurveto{\pgfqpoint{2.298925in}{1.498043in}}{\pgfqpoint{2.294534in}{1.487444in}}{\pgfqpoint{2.294534in}{1.476394in}}%
\pgfpathcurveto{\pgfqpoint{2.294534in}{1.465344in}}{\pgfqpoint{2.298925in}{1.454745in}}{\pgfqpoint{2.306738in}{1.446931in}}%
\pgfpathcurveto{\pgfqpoint{2.314552in}{1.439118in}}{\pgfqpoint{2.325151in}{1.434727in}}{\pgfqpoint{2.336201in}{1.434727in}}%
\pgfpathclose%
\pgfusepath{stroke,fill}%
\end{pgfscope}%
\begin{pgfscope}%
\pgfpathrectangle{\pgfqpoint{0.750000in}{0.375000in}}{\pgfqpoint{4.650000in}{2.265000in}}%
\pgfusepath{clip}%
\pgfsetbuttcap%
\pgfsetroundjoin%
\definecolor{currentfill}{rgb}{1.000000,0.498039,0.054902}%
\pgfsetfillcolor{currentfill}%
\pgfsetlinewidth{1.003750pt}%
\definecolor{currentstroke}{rgb}{1.000000,0.498039,0.054902}%
\pgfsetstrokecolor{currentstroke}%
\pgfsetdash{}{0pt}%
\pgfpathmoveto{\pgfqpoint{2.077564in}{1.450177in}}%
\pgfpathcurveto{\pgfqpoint{2.088614in}{1.450177in}}{\pgfqpoint{2.099213in}{1.454568in}}{\pgfqpoint{2.107027in}{1.462381in}}%
\pgfpathcurveto{\pgfqpoint{2.114841in}{1.470195in}}{\pgfqpoint{2.119231in}{1.480794in}}{\pgfqpoint{2.119231in}{1.491844in}}%
\pgfpathcurveto{\pgfqpoint{2.119231in}{1.502894in}}{\pgfqpoint{2.114841in}{1.513493in}}{\pgfqpoint{2.107027in}{1.521307in}}%
\pgfpathcurveto{\pgfqpoint{2.099213in}{1.529120in}}{\pgfqpoint{2.088614in}{1.533511in}}{\pgfqpoint{2.077564in}{1.533511in}}%
\pgfpathcurveto{\pgfqpoint{2.066514in}{1.533511in}}{\pgfqpoint{2.055915in}{1.529120in}}{\pgfqpoint{2.048101in}{1.521307in}}%
\pgfpathcurveto{\pgfqpoint{2.040288in}{1.513493in}}{\pgfqpoint{2.035898in}{1.502894in}}{\pgfqpoint{2.035898in}{1.491844in}}%
\pgfpathcurveto{\pgfqpoint{2.035898in}{1.480794in}}{\pgfqpoint{2.040288in}{1.470195in}}{\pgfqpoint{2.048101in}{1.462381in}}%
\pgfpathcurveto{\pgfqpoint{2.055915in}{1.454568in}}{\pgfqpoint{2.066514in}{1.450177in}}{\pgfqpoint{2.077564in}{1.450177in}}%
\pgfpathclose%
\pgfusepath{stroke,fill}%
\end{pgfscope}%
\begin{pgfscope}%
\pgfpathrectangle{\pgfqpoint{0.750000in}{0.375000in}}{\pgfqpoint{4.650000in}{2.265000in}}%
\pgfusepath{clip}%
\pgfsetbuttcap%
\pgfsetroundjoin%
\definecolor{currentfill}{rgb}{1.000000,0.498039,0.054902}%
\pgfsetfillcolor{currentfill}%
\pgfsetlinewidth{1.003750pt}%
\definecolor{currentstroke}{rgb}{1.000000,0.498039,0.054902}%
\pgfsetstrokecolor{currentstroke}%
\pgfsetdash{}{0pt}%
\pgfpathmoveto{\pgfqpoint{1.728405in}{1.457902in}}%
\pgfpathcurveto{\pgfqpoint{1.739455in}{1.457902in}}{\pgfqpoint{1.750054in}{1.462293in}}{\pgfqpoint{1.757867in}{1.470106in}}%
\pgfpathcurveto{\pgfqpoint{1.765681in}{1.477920in}}{\pgfqpoint{1.770071in}{1.488519in}}{\pgfqpoint{1.770071in}{1.499569in}}%
\pgfpathcurveto{\pgfqpoint{1.770071in}{1.510619in}}{\pgfqpoint{1.765681in}{1.521218in}}{\pgfqpoint{1.757867in}{1.529032in}}%
\pgfpathcurveto{\pgfqpoint{1.750054in}{1.536845in}}{\pgfqpoint{1.739455in}{1.541236in}}{\pgfqpoint{1.728405in}{1.541236in}}%
\pgfpathcurveto{\pgfqpoint{1.717355in}{1.541236in}}{\pgfqpoint{1.706756in}{1.536845in}}{\pgfqpoint{1.698942in}{1.529032in}}%
\pgfpathcurveto{\pgfqpoint{1.691128in}{1.521218in}}{\pgfqpoint{1.686738in}{1.510619in}}{\pgfqpoint{1.686738in}{1.499569in}}%
\pgfpathcurveto{\pgfqpoint{1.686738in}{1.488519in}}{\pgfqpoint{1.691128in}{1.477920in}}{\pgfqpoint{1.698942in}{1.470106in}}%
\pgfpathcurveto{\pgfqpoint{1.706756in}{1.462293in}}{\pgfqpoint{1.717355in}{1.457902in}}{\pgfqpoint{1.728405in}{1.457902in}}%
\pgfpathclose%
\pgfusepath{stroke,fill}%
\end{pgfscope}%
\begin{pgfscope}%
\pgfpathrectangle{\pgfqpoint{0.750000in}{0.375000in}}{\pgfqpoint{4.650000in}{2.265000in}}%
\pgfusepath{clip}%
\pgfsetbuttcap%
\pgfsetroundjoin%
\definecolor{currentfill}{rgb}{1.000000,0.498039,0.054902}%
\pgfsetfillcolor{currentfill}%
\pgfsetlinewidth{1.003750pt}%
\definecolor{currentstroke}{rgb}{1.000000,0.498039,0.054902}%
\pgfsetstrokecolor{currentstroke}%
\pgfsetdash{}{0pt}%
\pgfpathmoveto{\pgfqpoint{1.257039in}{1.461765in}}%
\pgfpathcurveto{\pgfqpoint{1.268089in}{1.461765in}}{\pgfqpoint{1.278688in}{1.466155in}}{\pgfqpoint{1.286502in}{1.473969in}}%
\pgfpathcurveto{\pgfqpoint{1.294316in}{1.481782in}}{\pgfqpoint{1.298706in}{1.492381in}}{\pgfqpoint{1.298706in}{1.503431in}}%
\pgfpathcurveto{\pgfqpoint{1.298706in}{1.514482in}}{\pgfqpoint{1.294316in}{1.525081in}}{\pgfqpoint{1.286502in}{1.532894in}}%
\pgfpathcurveto{\pgfqpoint{1.278688in}{1.540708in}}{\pgfqpoint{1.268089in}{1.545098in}}{\pgfqpoint{1.257039in}{1.545098in}}%
\pgfpathcurveto{\pgfqpoint{1.245989in}{1.545098in}}{\pgfqpoint{1.235390in}{1.540708in}}{\pgfqpoint{1.227577in}{1.532894in}}%
\pgfpathcurveto{\pgfqpoint{1.219763in}{1.525081in}}{\pgfqpoint{1.215373in}{1.514482in}}{\pgfqpoint{1.215373in}{1.503431in}}%
\pgfpathcurveto{\pgfqpoint{1.215373in}{1.492381in}}{\pgfqpoint{1.219763in}{1.481782in}}{\pgfqpoint{1.227577in}{1.473969in}}%
\pgfpathcurveto{\pgfqpoint{1.235390in}{1.466155in}}{\pgfqpoint{1.245989in}{1.461765in}}{\pgfqpoint{1.257039in}{1.461765in}}%
\pgfpathclose%
\pgfusepath{stroke,fill}%
\end{pgfscope}%
\begin{pgfscope}%
\pgfpathrectangle{\pgfqpoint{0.750000in}{0.375000in}}{\pgfqpoint{4.650000in}{2.265000in}}%
\pgfusepath{clip}%
\pgfsetbuttcap%
\pgfsetroundjoin%
\definecolor{currentfill}{rgb}{0.172549,0.627451,0.172549}%
\pgfsetfillcolor{currentfill}%
\pgfsetlinewidth{1.003750pt}%
\definecolor{currentstroke}{rgb}{0.172549,0.627451,0.172549}%
\pgfsetstrokecolor{currentstroke}%
\pgfsetdash{}{0pt}%
\pgfpathmoveto{\pgfqpoint{3.239962in}{0.476836in}}%
\pgfpathcurveto{\pgfqpoint{3.251012in}{0.476836in}}{\pgfqpoint{3.261611in}{0.481226in}}{\pgfqpoint{3.269424in}{0.489040in}}%
\pgfpathcurveto{\pgfqpoint{3.277238in}{0.496854in}}{\pgfqpoint{3.281628in}{0.507453in}}{\pgfqpoint{3.281628in}{0.518503in}}%
\pgfpathcurveto{\pgfqpoint{3.281628in}{0.529553in}}{\pgfqpoint{3.277238in}{0.540152in}}{\pgfqpoint{3.269424in}{0.547966in}}%
\pgfpathcurveto{\pgfqpoint{3.261611in}{0.555779in}}{\pgfqpoint{3.251012in}{0.560169in}}{\pgfqpoint{3.239962in}{0.560169in}}%
\pgfpathcurveto{\pgfqpoint{3.228911in}{0.560169in}}{\pgfqpoint{3.218312in}{0.555779in}}{\pgfqpoint{3.210499in}{0.547966in}}%
\pgfpathcurveto{\pgfqpoint{3.202685in}{0.540152in}}{\pgfqpoint{3.198295in}{0.529553in}}{\pgfqpoint{3.198295in}{0.518503in}}%
\pgfpathcurveto{\pgfqpoint{3.198295in}{0.507453in}}{\pgfqpoint{3.202685in}{0.496854in}}{\pgfqpoint{3.210499in}{0.489040in}}%
\pgfpathcurveto{\pgfqpoint{3.218312in}{0.481226in}}{\pgfqpoint{3.228911in}{0.476836in}}{\pgfqpoint{3.239962in}{0.476836in}}%
\pgfpathclose%
\pgfusepath{stroke,fill}%
\end{pgfscope}%
\begin{pgfscope}%
\pgfpathrectangle{\pgfqpoint{0.750000in}{0.375000in}}{\pgfqpoint{4.650000in}{2.265000in}}%
\pgfusepath{clip}%
\pgfsetbuttcap%
\pgfsetroundjoin%
\definecolor{currentfill}{rgb}{0.172549,0.627451,0.172549}%
\pgfsetfillcolor{currentfill}%
\pgfsetlinewidth{1.003750pt}%
\definecolor{currentstroke}{rgb}{0.172549,0.627451,0.172549}%
\pgfsetstrokecolor{currentstroke}%
\pgfsetdash{}{0pt}%
\pgfpathmoveto{\pgfqpoint{3.297641in}{0.971232in}}%
\pgfpathcurveto{\pgfqpoint{3.308691in}{0.971232in}}{\pgfqpoint{3.319290in}{0.975622in}}{\pgfqpoint{3.327104in}{0.983436in}}%
\pgfpathcurveto{\pgfqpoint{3.334917in}{0.991249in}}{\pgfqpoint{3.339308in}{1.001848in}}{\pgfqpoint{3.339308in}{1.012898in}}%
\pgfpathcurveto{\pgfqpoint{3.339308in}{1.023948in}}{\pgfqpoint{3.334917in}{1.034548in}}{\pgfqpoint{3.327104in}{1.042361in}}%
\pgfpathcurveto{\pgfqpoint{3.319290in}{1.050175in}}{\pgfqpoint{3.308691in}{1.054565in}}{\pgfqpoint{3.297641in}{1.054565in}}%
\pgfpathcurveto{\pgfqpoint{3.286591in}{1.054565in}}{\pgfqpoint{3.275992in}{1.050175in}}{\pgfqpoint{3.268178in}{1.042361in}}%
\pgfpathcurveto{\pgfqpoint{3.260365in}{1.034548in}}{\pgfqpoint{3.255974in}{1.023948in}}{\pgfqpoint{3.255974in}{1.012898in}}%
\pgfpathcurveto{\pgfqpoint{3.255974in}{1.001848in}}{\pgfqpoint{3.260365in}{0.991249in}}{\pgfqpoint{3.268178in}{0.983436in}}%
\pgfpathcurveto{\pgfqpoint{3.275992in}{0.975622in}}{\pgfqpoint{3.286591in}{0.971232in}}{\pgfqpoint{3.297641in}{0.971232in}}%
\pgfpathclose%
\pgfusepath{stroke,fill}%
\end{pgfscope}%
\begin{pgfscope}%
\pgfpathrectangle{\pgfqpoint{0.750000in}{0.375000in}}{\pgfqpoint{4.650000in}{2.265000in}}%
\pgfusepath{clip}%
\pgfsetbuttcap%
\pgfsetroundjoin%
\definecolor{currentfill}{rgb}{0.172549,0.627451,0.172549}%
\pgfsetfillcolor{currentfill}%
\pgfsetlinewidth{1.003750pt}%
\definecolor{currentstroke}{rgb}{0.172549,0.627451,0.172549}%
\pgfsetstrokecolor{currentstroke}%
\pgfsetdash{}{0pt}%
\pgfpathmoveto{\pgfqpoint{3.375508in}{1.218429in}}%
\pgfpathcurveto{\pgfqpoint{3.386558in}{1.218429in}}{\pgfqpoint{3.397157in}{1.222820in}}{\pgfqpoint{3.404971in}{1.230633in}}%
\pgfpathcurveto{\pgfqpoint{3.412785in}{1.238447in}}{\pgfqpoint{3.417175in}{1.249046in}}{\pgfqpoint{3.417175in}{1.260096in}}%
\pgfpathcurveto{\pgfqpoint{3.417175in}{1.271146in}}{\pgfqpoint{3.412785in}{1.281745in}}{\pgfqpoint{3.404971in}{1.289559in}}%
\pgfpathcurveto{\pgfqpoint{3.397157in}{1.297373in}}{\pgfqpoint{3.386558in}{1.301763in}}{\pgfqpoint{3.375508in}{1.301763in}}%
\pgfpathcurveto{\pgfqpoint{3.364458in}{1.301763in}}{\pgfqpoint{3.353859in}{1.297373in}}{\pgfqpoint{3.346046in}{1.289559in}}%
\pgfpathcurveto{\pgfqpoint{3.338232in}{1.281745in}}{\pgfqpoint{3.333842in}{1.271146in}}{\pgfqpoint{3.333842in}{1.260096in}}%
\pgfpathcurveto{\pgfqpoint{3.333842in}{1.249046in}}{\pgfqpoint{3.338232in}{1.238447in}}{\pgfqpoint{3.346046in}{1.230633in}}%
\pgfpathcurveto{\pgfqpoint{3.353859in}{1.222820in}}{\pgfqpoint{3.364458in}{1.218429in}}{\pgfqpoint{3.375508in}{1.218429in}}%
\pgfpathclose%
\pgfusepath{stroke,fill}%
\end{pgfscope}%
\begin{pgfscope}%
\pgfpathrectangle{\pgfqpoint{0.750000in}{0.375000in}}{\pgfqpoint{4.650000in}{2.265000in}}%
\pgfusepath{clip}%
\pgfsetbuttcap%
\pgfsetroundjoin%
\definecolor{currentfill}{rgb}{0.172549,0.627451,0.172549}%
\pgfsetfillcolor{currentfill}%
\pgfsetlinewidth{1.003750pt}%
\definecolor{currentstroke}{rgb}{0.172549,0.627451,0.172549}%
\pgfsetstrokecolor{currentstroke}%
\pgfsetdash{}{0pt}%
\pgfpathmoveto{\pgfqpoint{3.480629in}{1.342028in}}%
\pgfpathcurveto{\pgfqpoint{3.491679in}{1.342028in}}{\pgfqpoint{3.502278in}{1.346419in}}{\pgfqpoint{3.510092in}{1.354232in}}%
\pgfpathcurveto{\pgfqpoint{3.517906in}{1.362046in}}{\pgfqpoint{3.522296in}{1.372645in}}{\pgfqpoint{3.522296in}{1.383695in}}%
\pgfpathcurveto{\pgfqpoint{3.522296in}{1.394745in}}{\pgfqpoint{3.517906in}{1.405344in}}{\pgfqpoint{3.510092in}{1.413158in}}%
\pgfpathcurveto{\pgfqpoint{3.502278in}{1.420971in}}{\pgfqpoint{3.491679in}{1.425362in}}{\pgfqpoint{3.480629in}{1.425362in}}%
\pgfpathcurveto{\pgfqpoint{3.469579in}{1.425362in}}{\pgfqpoint{3.458980in}{1.420971in}}{\pgfqpoint{3.451166in}{1.413158in}}%
\pgfpathcurveto{\pgfqpoint{3.443353in}{1.405344in}}{\pgfqpoint{3.438962in}{1.394745in}}{\pgfqpoint{3.438962in}{1.383695in}}%
\pgfpathcurveto{\pgfqpoint{3.438962in}{1.372645in}}{\pgfqpoint{3.443353in}{1.362046in}}{\pgfqpoint{3.451166in}{1.354232in}}%
\pgfpathcurveto{\pgfqpoint{3.458980in}{1.346419in}}{\pgfqpoint{3.469579in}{1.342028in}}{\pgfqpoint{3.480629in}{1.342028in}}%
\pgfpathclose%
\pgfusepath{stroke,fill}%
\end{pgfscope}%
\begin{pgfscope}%
\pgfpathrectangle{\pgfqpoint{0.750000in}{0.375000in}}{\pgfqpoint{4.650000in}{2.265000in}}%
\pgfusepath{clip}%
\pgfsetbuttcap%
\pgfsetroundjoin%
\definecolor{currentfill}{rgb}{0.172549,0.627451,0.172549}%
\pgfsetfillcolor{currentfill}%
\pgfsetlinewidth{1.003750pt}%
\definecolor{currentstroke}{rgb}{0.172549,0.627451,0.172549}%
\pgfsetstrokecolor{currentstroke}%
\pgfsetdash{}{0pt}%
\pgfpathmoveto{\pgfqpoint{3.622542in}{1.403828in}}%
\pgfpathcurveto{\pgfqpoint{3.633592in}{1.403828in}}{\pgfqpoint{3.644191in}{1.408218in}}{\pgfqpoint{3.652005in}{1.416032in}}%
\pgfpathcurveto{\pgfqpoint{3.659819in}{1.423845in}}{\pgfqpoint{3.664209in}{1.434444in}}{\pgfqpoint{3.664209in}{1.445494in}}%
\pgfpathcurveto{\pgfqpoint{3.664209in}{1.456545in}}{\pgfqpoint{3.659819in}{1.467144in}}{\pgfqpoint{3.652005in}{1.474957in}}%
\pgfpathcurveto{\pgfqpoint{3.644191in}{1.482771in}}{\pgfqpoint{3.633592in}{1.487161in}}{\pgfqpoint{3.622542in}{1.487161in}}%
\pgfpathcurveto{\pgfqpoint{3.611492in}{1.487161in}}{\pgfqpoint{3.600893in}{1.482771in}}{\pgfqpoint{3.593080in}{1.474957in}}%
\pgfpathcurveto{\pgfqpoint{3.585266in}{1.467144in}}{\pgfqpoint{3.580876in}{1.456545in}}{\pgfqpoint{3.580876in}{1.445494in}}%
\pgfpathcurveto{\pgfqpoint{3.580876in}{1.434444in}}{\pgfqpoint{3.585266in}{1.423845in}}{\pgfqpoint{3.593080in}{1.416032in}}%
\pgfpathcurveto{\pgfqpoint{3.600893in}{1.408218in}}{\pgfqpoint{3.611492in}{1.403828in}}{\pgfqpoint{3.622542in}{1.403828in}}%
\pgfpathclose%
\pgfusepath{stroke,fill}%
\end{pgfscope}%
\begin{pgfscope}%
\pgfpathrectangle{\pgfqpoint{0.750000in}{0.375000in}}{\pgfqpoint{4.650000in}{2.265000in}}%
\pgfusepath{clip}%
\pgfsetbuttcap%
\pgfsetroundjoin%
\definecolor{currentfill}{rgb}{0.172549,0.627451,0.172549}%
\pgfsetfillcolor{currentfill}%
\pgfsetlinewidth{1.003750pt}%
\definecolor{currentstroke}{rgb}{0.172549,0.627451,0.172549}%
\pgfsetstrokecolor{currentstroke}%
\pgfsetdash{}{0pt}%
\pgfpathmoveto{\pgfqpoint{3.814125in}{1.434727in}}%
\pgfpathcurveto{\pgfqpoint{3.825175in}{1.434727in}}{\pgfqpoint{3.835774in}{1.439118in}}{\pgfqpoint{3.843588in}{1.446931in}}%
\pgfpathcurveto{\pgfqpoint{3.851401in}{1.454745in}}{\pgfqpoint{3.855792in}{1.465344in}}{\pgfqpoint{3.855792in}{1.476394in}}%
\pgfpathcurveto{\pgfqpoint{3.855792in}{1.487444in}}{\pgfqpoint{3.851401in}{1.498043in}}{\pgfqpoint{3.843588in}{1.505857in}}%
\pgfpathcurveto{\pgfqpoint{3.835774in}{1.513671in}}{\pgfqpoint{3.825175in}{1.518061in}}{\pgfqpoint{3.814125in}{1.518061in}}%
\pgfpathcurveto{\pgfqpoint{3.803075in}{1.518061in}}{\pgfqpoint{3.792476in}{1.513671in}}{\pgfqpoint{3.784662in}{1.505857in}}%
\pgfpathcurveto{\pgfqpoint{3.776849in}{1.498043in}}{\pgfqpoint{3.772458in}{1.487444in}}{\pgfqpoint{3.772458in}{1.476394in}}%
\pgfpathcurveto{\pgfqpoint{3.772458in}{1.465344in}}{\pgfqpoint{3.776849in}{1.454745in}}{\pgfqpoint{3.784662in}{1.446931in}}%
\pgfpathcurveto{\pgfqpoint{3.792476in}{1.439118in}}{\pgfqpoint{3.803075in}{1.434727in}}{\pgfqpoint{3.814125in}{1.434727in}}%
\pgfpathclose%
\pgfusepath{stroke,fill}%
\end{pgfscope}%
\begin{pgfscope}%
\pgfpathrectangle{\pgfqpoint{0.750000in}{0.375000in}}{\pgfqpoint{4.650000in}{2.265000in}}%
\pgfusepath{clip}%
\pgfsetbuttcap%
\pgfsetroundjoin%
\definecolor{currentfill}{rgb}{0.172549,0.627451,0.172549}%
\pgfsetfillcolor{currentfill}%
\pgfsetlinewidth{1.003750pt}%
\definecolor{currentstroke}{rgb}{0.172549,0.627451,0.172549}%
\pgfsetstrokecolor{currentstroke}%
\pgfsetdash{}{0pt}%
\pgfpathmoveto{\pgfqpoint{4.072762in}{1.450177in}}%
\pgfpathcurveto{\pgfqpoint{4.083812in}{1.450177in}}{\pgfqpoint{4.094411in}{1.454568in}}{\pgfqpoint{4.102225in}{1.462381in}}%
\pgfpathcurveto{\pgfqpoint{4.110038in}{1.470195in}}{\pgfqpoint{4.114428in}{1.480794in}}{\pgfqpoint{4.114428in}{1.491844in}}%
\pgfpathcurveto{\pgfqpoint{4.114428in}{1.502894in}}{\pgfqpoint{4.110038in}{1.513493in}}{\pgfqpoint{4.102225in}{1.521307in}}%
\pgfpathcurveto{\pgfqpoint{4.094411in}{1.529120in}}{\pgfqpoint{4.083812in}{1.533511in}}{\pgfqpoint{4.072762in}{1.533511in}}%
\pgfpathcurveto{\pgfqpoint{4.061712in}{1.533511in}}{\pgfqpoint{4.051113in}{1.529120in}}{\pgfqpoint{4.043299in}{1.521307in}}%
\pgfpathcurveto{\pgfqpoint{4.035485in}{1.513493in}}{\pgfqpoint{4.031095in}{1.502894in}}{\pgfqpoint{4.031095in}{1.491844in}}%
\pgfpathcurveto{\pgfqpoint{4.031095in}{1.480794in}}{\pgfqpoint{4.035485in}{1.470195in}}{\pgfqpoint{4.043299in}{1.462381in}}%
\pgfpathcurveto{\pgfqpoint{4.051113in}{1.454568in}}{\pgfqpoint{4.061712in}{1.450177in}}{\pgfqpoint{4.072762in}{1.450177in}}%
\pgfpathclose%
\pgfusepath{stroke,fill}%
\end{pgfscope}%
\begin{pgfscope}%
\pgfpathrectangle{\pgfqpoint{0.750000in}{0.375000in}}{\pgfqpoint{4.650000in}{2.265000in}}%
\pgfusepath{clip}%
\pgfsetbuttcap%
\pgfsetroundjoin%
\definecolor{currentfill}{rgb}{0.172549,0.627451,0.172549}%
\pgfsetfillcolor{currentfill}%
\pgfsetlinewidth{1.003750pt}%
\definecolor{currentstroke}{rgb}{0.172549,0.627451,0.172549}%
\pgfsetstrokecolor{currentstroke}%
\pgfsetdash{}{0pt}%
\pgfpathmoveto{\pgfqpoint{4.421921in}{1.457902in}}%
\pgfpathcurveto{\pgfqpoint{4.432971in}{1.457902in}}{\pgfqpoint{4.443570in}{1.462293in}}{\pgfqpoint{4.451384in}{1.470106in}}%
\pgfpathcurveto{\pgfqpoint{4.459198in}{1.477920in}}{\pgfqpoint{4.463588in}{1.488519in}}{\pgfqpoint{4.463588in}{1.499569in}}%
\pgfpathcurveto{\pgfqpoint{4.463588in}{1.510619in}}{\pgfqpoint{4.459198in}{1.521218in}}{\pgfqpoint{4.451384in}{1.529032in}}%
\pgfpathcurveto{\pgfqpoint{4.443570in}{1.536845in}}{\pgfqpoint{4.432971in}{1.541236in}}{\pgfqpoint{4.421921in}{1.541236in}}%
\pgfpathcurveto{\pgfqpoint{4.410871in}{1.541236in}}{\pgfqpoint{4.400272in}{1.536845in}}{\pgfqpoint{4.392459in}{1.529032in}}%
\pgfpathcurveto{\pgfqpoint{4.384645in}{1.521218in}}{\pgfqpoint{4.380255in}{1.510619in}}{\pgfqpoint{4.380255in}{1.499569in}}%
\pgfpathcurveto{\pgfqpoint{4.380255in}{1.488519in}}{\pgfqpoint{4.384645in}{1.477920in}}{\pgfqpoint{4.392459in}{1.470106in}}%
\pgfpathcurveto{\pgfqpoint{4.400272in}{1.462293in}}{\pgfqpoint{4.410871in}{1.457902in}}{\pgfqpoint{4.421921in}{1.457902in}}%
\pgfpathclose%
\pgfusepath{stroke,fill}%
\end{pgfscope}%
\begin{pgfscope}%
\pgfpathrectangle{\pgfqpoint{0.750000in}{0.375000in}}{\pgfqpoint{4.650000in}{2.265000in}}%
\pgfusepath{clip}%
\pgfsetbuttcap%
\pgfsetroundjoin%
\definecolor{currentfill}{rgb}{0.172549,0.627451,0.172549}%
\pgfsetfillcolor{currentfill}%
\pgfsetlinewidth{1.003750pt}%
\definecolor{currentstroke}{rgb}{0.172549,0.627451,0.172549}%
\pgfsetstrokecolor{currentstroke}%
\pgfsetdash{}{0pt}%
\pgfpathmoveto{\pgfqpoint{4.893287in}{1.461765in}}%
\pgfpathcurveto{\pgfqpoint{4.904337in}{1.461765in}}{\pgfqpoint{4.914936in}{1.466155in}}{\pgfqpoint{4.922750in}{1.473969in}}%
\pgfpathcurveto{\pgfqpoint{4.930563in}{1.481782in}}{\pgfqpoint{4.934953in}{1.492381in}}{\pgfqpoint{4.934953in}{1.503431in}}%
\pgfpathcurveto{\pgfqpoint{4.934953in}{1.514482in}}{\pgfqpoint{4.930563in}{1.525081in}}{\pgfqpoint{4.922750in}{1.532894in}}%
\pgfpathcurveto{\pgfqpoint{4.914936in}{1.540708in}}{\pgfqpoint{4.904337in}{1.545098in}}{\pgfqpoint{4.893287in}{1.545098in}}%
\pgfpathcurveto{\pgfqpoint{4.882237in}{1.545098in}}{\pgfqpoint{4.871638in}{1.540708in}}{\pgfqpoint{4.863824in}{1.532894in}}%
\pgfpathcurveto{\pgfqpoint{4.856010in}{1.525081in}}{\pgfqpoint{4.851620in}{1.514482in}}{\pgfqpoint{4.851620in}{1.503431in}}%
\pgfpathcurveto{\pgfqpoint{4.851620in}{1.492381in}}{\pgfqpoint{4.856010in}{1.481782in}}{\pgfqpoint{4.863824in}{1.473969in}}%
\pgfpathcurveto{\pgfqpoint{4.871638in}{1.466155in}}{\pgfqpoint{4.882237in}{1.461765in}}{\pgfqpoint{4.893287in}{1.461765in}}%
\pgfpathclose%
\pgfusepath{stroke,fill}%
\end{pgfscope}%
\begin{pgfscope}%
\pgfpathrectangle{\pgfqpoint{0.750000in}{0.375000in}}{\pgfqpoint{4.650000in}{2.265000in}}%
\pgfusepath{clip}%
\pgfsetbuttcap%
\pgfsetroundjoin%
\definecolor{currentfill}{rgb}{0.839216,0.152941,0.156863}%
\pgfsetfillcolor{currentfill}%
\pgfsetlinewidth{1.003750pt}%
\definecolor{currentstroke}{rgb}{0.839216,0.152941,0.156863}%
\pgfsetstrokecolor{currentstroke}%
\pgfsetdash{}{0pt}%
\pgfpathmoveto{\pgfqpoint{2.910365in}{2.454418in}}%
\pgfpathcurveto{\pgfqpoint{2.921415in}{2.454418in}}{\pgfqpoint{2.932014in}{2.458809in}}{\pgfqpoint{2.939827in}{2.466622in}}%
\pgfpathcurveto{\pgfqpoint{2.947641in}{2.474436in}}{\pgfqpoint{2.952031in}{2.485035in}}{\pgfqpoint{2.952031in}{2.496085in}}%
\pgfpathcurveto{\pgfqpoint{2.952031in}{2.507135in}}{\pgfqpoint{2.947641in}{2.517734in}}{\pgfqpoint{2.939827in}{2.525548in}}%
\pgfpathcurveto{\pgfqpoint{2.932014in}{2.533361in}}{\pgfqpoint{2.921415in}{2.537752in}}{\pgfqpoint{2.910365in}{2.537752in}}%
\pgfpathcurveto{\pgfqpoint{2.899314in}{2.537752in}}{\pgfqpoint{2.888715in}{2.533361in}}{\pgfqpoint{2.880902in}{2.525548in}}%
\pgfpathcurveto{\pgfqpoint{2.873088in}{2.517734in}}{\pgfqpoint{2.868698in}{2.507135in}}{\pgfqpoint{2.868698in}{2.496085in}}%
\pgfpathcurveto{\pgfqpoint{2.868698in}{2.485035in}}{\pgfqpoint{2.873088in}{2.474436in}}{\pgfqpoint{2.880902in}{2.466622in}}%
\pgfpathcurveto{\pgfqpoint{2.888715in}{2.458809in}}{\pgfqpoint{2.899314in}{2.454418in}}{\pgfqpoint{2.910365in}{2.454418in}}%
\pgfpathclose%
\pgfusepath{stroke,fill}%
\end{pgfscope}%
\begin{pgfscope}%
\pgfpathrectangle{\pgfqpoint{0.750000in}{0.375000in}}{\pgfqpoint{4.650000in}{2.265000in}}%
\pgfusepath{clip}%
\pgfsetbuttcap%
\pgfsetroundjoin%
\definecolor{currentfill}{rgb}{0.839216,0.152941,0.156863}%
\pgfsetfillcolor{currentfill}%
\pgfsetlinewidth{1.003750pt}%
\definecolor{currentstroke}{rgb}{0.839216,0.152941,0.156863}%
\pgfsetstrokecolor{currentstroke}%
\pgfsetdash{}{0pt}%
\pgfpathmoveto{\pgfqpoint{2.852685in}{1.960023in}}%
\pgfpathcurveto{\pgfqpoint{2.863735in}{1.960023in}}{\pgfqpoint{2.874334in}{1.964413in}}{\pgfqpoint{2.882148in}{1.972227in}}%
\pgfpathcurveto{\pgfqpoint{2.889961in}{1.980040in}}{\pgfqpoint{2.894352in}{1.990639in}}{\pgfqpoint{2.894352in}{2.001689in}}%
\pgfpathcurveto{\pgfqpoint{2.894352in}{2.012740in}}{\pgfqpoint{2.889961in}{2.023339in}}{\pgfqpoint{2.882148in}{2.031152in}}%
\pgfpathcurveto{\pgfqpoint{2.874334in}{2.038966in}}{\pgfqpoint{2.863735in}{2.043356in}}{\pgfqpoint{2.852685in}{2.043356in}}%
\pgfpathcurveto{\pgfqpoint{2.841635in}{2.043356in}}{\pgfqpoint{2.831036in}{2.038966in}}{\pgfqpoint{2.823222in}{2.031152in}}%
\pgfpathcurveto{\pgfqpoint{2.815409in}{2.023339in}}{\pgfqpoint{2.811018in}{2.012740in}}{\pgfqpoint{2.811018in}{2.001689in}}%
\pgfpathcurveto{\pgfqpoint{2.811018in}{1.990639in}}{\pgfqpoint{2.815409in}{1.980040in}}{\pgfqpoint{2.823222in}{1.972227in}}%
\pgfpathcurveto{\pgfqpoint{2.831036in}{1.964413in}}{\pgfqpoint{2.841635in}{1.960023in}}{\pgfqpoint{2.852685in}{1.960023in}}%
\pgfpathclose%
\pgfusepath{stroke,fill}%
\end{pgfscope}%
\begin{pgfscope}%
\pgfpathrectangle{\pgfqpoint{0.750000in}{0.375000in}}{\pgfqpoint{4.650000in}{2.265000in}}%
\pgfusepath{clip}%
\pgfsetbuttcap%
\pgfsetroundjoin%
\definecolor{currentfill}{rgb}{0.839216,0.152941,0.156863}%
\pgfsetfillcolor{currentfill}%
\pgfsetlinewidth{1.003750pt}%
\definecolor{currentstroke}{rgb}{0.839216,0.152941,0.156863}%
\pgfsetstrokecolor{currentstroke}%
\pgfsetdash{}{0pt}%
\pgfpathmoveto{\pgfqpoint{2.774818in}{1.712825in}}%
\pgfpathcurveto{\pgfqpoint{2.785868in}{1.712825in}}{\pgfqpoint{2.796467in}{1.717215in}}{\pgfqpoint{2.804281in}{1.725029in}}%
\pgfpathcurveto{\pgfqpoint{2.812094in}{1.732842in}}{\pgfqpoint{2.816484in}{1.743442in}}{\pgfqpoint{2.816484in}{1.754492in}}%
\pgfpathcurveto{\pgfqpoint{2.816484in}{1.765542in}}{\pgfqpoint{2.812094in}{1.776141in}}{\pgfqpoint{2.804281in}{1.783954in}}%
\pgfpathcurveto{\pgfqpoint{2.796467in}{1.791768in}}{\pgfqpoint{2.785868in}{1.796158in}}{\pgfqpoint{2.774818in}{1.796158in}}%
\pgfpathcurveto{\pgfqpoint{2.763768in}{1.796158in}}{\pgfqpoint{2.753169in}{1.791768in}}{\pgfqpoint{2.745355in}{1.783954in}}%
\pgfpathcurveto{\pgfqpoint{2.737541in}{1.776141in}}{\pgfqpoint{2.733151in}{1.765542in}}{\pgfqpoint{2.733151in}{1.754492in}}%
\pgfpathcurveto{\pgfqpoint{2.733151in}{1.743442in}}{\pgfqpoint{2.737541in}{1.732842in}}{\pgfqpoint{2.745355in}{1.725029in}}%
\pgfpathcurveto{\pgfqpoint{2.753169in}{1.717215in}}{\pgfqpoint{2.763768in}{1.712825in}}{\pgfqpoint{2.774818in}{1.712825in}}%
\pgfpathclose%
\pgfusepath{stroke,fill}%
\end{pgfscope}%
\begin{pgfscope}%
\pgfpathrectangle{\pgfqpoint{0.750000in}{0.375000in}}{\pgfqpoint{4.650000in}{2.265000in}}%
\pgfusepath{clip}%
\pgfsetbuttcap%
\pgfsetroundjoin%
\definecolor{currentfill}{rgb}{0.839216,0.152941,0.156863}%
\pgfsetfillcolor{currentfill}%
\pgfsetlinewidth{1.003750pt}%
\definecolor{currentstroke}{rgb}{0.839216,0.152941,0.156863}%
\pgfsetstrokecolor{currentstroke}%
\pgfsetdash{}{0pt}%
\pgfpathmoveto{\pgfqpoint{2.669697in}{1.589226in}}%
\pgfpathcurveto{\pgfqpoint{2.680747in}{1.589226in}}{\pgfqpoint{2.691346in}{1.593616in}}{\pgfqpoint{2.699160in}{1.601430in}}%
\pgfpathcurveto{\pgfqpoint{2.706973in}{1.609244in}}{\pgfqpoint{2.711364in}{1.619843in}}{\pgfqpoint{2.711364in}{1.630893in}}%
\pgfpathcurveto{\pgfqpoint{2.711364in}{1.641943in}}{\pgfqpoint{2.706973in}{1.652542in}}{\pgfqpoint{2.699160in}{1.660356in}}%
\pgfpathcurveto{\pgfqpoint{2.691346in}{1.668169in}}{\pgfqpoint{2.680747in}{1.672559in}}{\pgfqpoint{2.669697in}{1.672559in}}%
\pgfpathcurveto{\pgfqpoint{2.658647in}{1.672559in}}{\pgfqpoint{2.648048in}{1.668169in}}{\pgfqpoint{2.640234in}{1.660356in}}%
\pgfpathcurveto{\pgfqpoint{2.632420in}{1.652542in}}{\pgfqpoint{2.628030in}{1.641943in}}{\pgfqpoint{2.628030in}{1.630893in}}%
\pgfpathcurveto{\pgfqpoint{2.628030in}{1.619843in}}{\pgfqpoint{2.632420in}{1.609244in}}{\pgfqpoint{2.640234in}{1.601430in}}%
\pgfpathcurveto{\pgfqpoint{2.648048in}{1.593616in}}{\pgfqpoint{2.658647in}{1.589226in}}{\pgfqpoint{2.669697in}{1.589226in}}%
\pgfpathclose%
\pgfusepath{stroke,fill}%
\end{pgfscope}%
\begin{pgfscope}%
\pgfpathrectangle{\pgfqpoint{0.750000in}{0.375000in}}{\pgfqpoint{4.650000in}{2.265000in}}%
\pgfusepath{clip}%
\pgfsetbuttcap%
\pgfsetroundjoin%
\definecolor{currentfill}{rgb}{0.839216,0.152941,0.156863}%
\pgfsetfillcolor{currentfill}%
\pgfsetlinewidth{1.003750pt}%
\definecolor{currentstroke}{rgb}{0.839216,0.152941,0.156863}%
\pgfsetstrokecolor{currentstroke}%
\pgfsetdash{}{0pt}%
\pgfpathmoveto{\pgfqpoint{2.527784in}{1.527427in}}%
\pgfpathcurveto{\pgfqpoint{2.538834in}{1.527427in}}{\pgfqpoint{2.549433in}{1.531817in}}{\pgfqpoint{2.557247in}{1.539631in}}%
\pgfpathcurveto{\pgfqpoint{2.565060in}{1.547444in}}{\pgfqpoint{2.569450in}{1.558043in}}{\pgfqpoint{2.569450in}{1.569093in}}%
\pgfpathcurveto{\pgfqpoint{2.569450in}{1.580143in}}{\pgfqpoint{2.565060in}{1.590742in}}{\pgfqpoint{2.557247in}{1.598556in}}%
\pgfpathcurveto{\pgfqpoint{2.549433in}{1.606370in}}{\pgfqpoint{2.538834in}{1.610760in}}{\pgfqpoint{2.527784in}{1.610760in}}%
\pgfpathcurveto{\pgfqpoint{2.516734in}{1.610760in}}{\pgfqpoint{2.506135in}{1.606370in}}{\pgfqpoint{2.498321in}{1.598556in}}%
\pgfpathcurveto{\pgfqpoint{2.490507in}{1.590742in}}{\pgfqpoint{2.486117in}{1.580143in}}{\pgfqpoint{2.486117in}{1.569093in}}%
\pgfpathcurveto{\pgfqpoint{2.486117in}{1.558043in}}{\pgfqpoint{2.490507in}{1.547444in}}{\pgfqpoint{2.498321in}{1.539631in}}%
\pgfpathcurveto{\pgfqpoint{2.506135in}{1.531817in}}{\pgfqpoint{2.516734in}{1.527427in}}{\pgfqpoint{2.527784in}{1.527427in}}%
\pgfpathclose%
\pgfusepath{stroke,fill}%
\end{pgfscope}%
\begin{pgfscope}%
\pgfpathrectangle{\pgfqpoint{0.750000in}{0.375000in}}{\pgfqpoint{4.650000in}{2.265000in}}%
\pgfusepath{clip}%
\pgfsetbuttcap%
\pgfsetroundjoin%
\definecolor{currentfill}{rgb}{0.839216,0.152941,0.156863}%
\pgfsetfillcolor{currentfill}%
\pgfsetlinewidth{1.003750pt}%
\definecolor{currentstroke}{rgb}{0.839216,0.152941,0.156863}%
\pgfsetstrokecolor{currentstroke}%
\pgfsetdash{}{0pt}%
\pgfpathmoveto{\pgfqpoint{2.336201in}{1.496527in}}%
\pgfpathcurveto{\pgfqpoint{2.347251in}{1.496527in}}{\pgfqpoint{2.357850in}{1.500917in}}{\pgfqpoint{2.365664in}{1.508731in}}%
\pgfpathcurveto{\pgfqpoint{2.373477in}{1.516544in}}{\pgfqpoint{2.377868in}{1.527143in}}{\pgfqpoint{2.377868in}{1.538194in}}%
\pgfpathcurveto{\pgfqpoint{2.377868in}{1.549244in}}{\pgfqpoint{2.373477in}{1.559843in}}{\pgfqpoint{2.365664in}{1.567656in}}%
\pgfpathcurveto{\pgfqpoint{2.357850in}{1.575470in}}{\pgfqpoint{2.347251in}{1.579860in}}{\pgfqpoint{2.336201in}{1.579860in}}%
\pgfpathcurveto{\pgfqpoint{2.325151in}{1.579860in}}{\pgfqpoint{2.314552in}{1.575470in}}{\pgfqpoint{2.306738in}{1.567656in}}%
\pgfpathcurveto{\pgfqpoint{2.298925in}{1.559843in}}{\pgfqpoint{2.294534in}{1.549244in}}{\pgfqpoint{2.294534in}{1.538194in}}%
\pgfpathcurveto{\pgfqpoint{2.294534in}{1.527143in}}{\pgfqpoint{2.298925in}{1.516544in}}{\pgfqpoint{2.306738in}{1.508731in}}%
\pgfpathcurveto{\pgfqpoint{2.314552in}{1.500917in}}{\pgfqpoint{2.325151in}{1.496527in}}{\pgfqpoint{2.336201in}{1.496527in}}%
\pgfpathclose%
\pgfusepath{stroke,fill}%
\end{pgfscope}%
\begin{pgfscope}%
\pgfpathrectangle{\pgfqpoint{0.750000in}{0.375000in}}{\pgfqpoint{4.650000in}{2.265000in}}%
\pgfusepath{clip}%
\pgfsetbuttcap%
\pgfsetroundjoin%
\definecolor{currentfill}{rgb}{0.839216,0.152941,0.156863}%
\pgfsetfillcolor{currentfill}%
\pgfsetlinewidth{1.003750pt}%
\definecolor{currentstroke}{rgb}{0.839216,0.152941,0.156863}%
\pgfsetstrokecolor{currentstroke}%
\pgfsetdash{}{0pt}%
\pgfpathmoveto{\pgfqpoint{2.077564in}{1.481077in}}%
\pgfpathcurveto{\pgfqpoint{2.088614in}{1.481077in}}{\pgfqpoint{2.099213in}{1.485467in}}{\pgfqpoint{2.107027in}{1.493281in}}%
\pgfpathcurveto{\pgfqpoint{2.114841in}{1.501095in}}{\pgfqpoint{2.119231in}{1.511694in}}{\pgfqpoint{2.119231in}{1.522744in}}%
\pgfpathcurveto{\pgfqpoint{2.119231in}{1.533794in}}{\pgfqpoint{2.114841in}{1.544393in}}{\pgfqpoint{2.107027in}{1.552207in}}%
\pgfpathcurveto{\pgfqpoint{2.099213in}{1.560020in}}{\pgfqpoint{2.088614in}{1.564410in}}{\pgfqpoint{2.077564in}{1.564410in}}%
\pgfpathcurveto{\pgfqpoint{2.066514in}{1.564410in}}{\pgfqpoint{2.055915in}{1.560020in}}{\pgfqpoint{2.048101in}{1.552207in}}%
\pgfpathcurveto{\pgfqpoint{2.040288in}{1.544393in}}{\pgfqpoint{2.035898in}{1.533794in}}{\pgfqpoint{2.035898in}{1.522744in}}%
\pgfpathcurveto{\pgfqpoint{2.035898in}{1.511694in}}{\pgfqpoint{2.040288in}{1.501095in}}{\pgfqpoint{2.048101in}{1.493281in}}%
\pgfpathcurveto{\pgfqpoint{2.055915in}{1.485467in}}{\pgfqpoint{2.066514in}{1.481077in}}{\pgfqpoint{2.077564in}{1.481077in}}%
\pgfpathclose%
\pgfusepath{stroke,fill}%
\end{pgfscope}%
\begin{pgfscope}%
\pgfpathrectangle{\pgfqpoint{0.750000in}{0.375000in}}{\pgfqpoint{4.650000in}{2.265000in}}%
\pgfusepath{clip}%
\pgfsetbuttcap%
\pgfsetroundjoin%
\definecolor{currentfill}{rgb}{0.839216,0.152941,0.156863}%
\pgfsetfillcolor{currentfill}%
\pgfsetlinewidth{1.003750pt}%
\definecolor{currentstroke}{rgb}{0.839216,0.152941,0.156863}%
\pgfsetstrokecolor{currentstroke}%
\pgfsetdash{}{0pt}%
\pgfpathmoveto{\pgfqpoint{1.728405in}{1.473352in}}%
\pgfpathcurveto{\pgfqpoint{1.739455in}{1.473352in}}{\pgfqpoint{1.750054in}{1.477742in}}{\pgfqpoint{1.757867in}{1.485556in}}%
\pgfpathcurveto{\pgfqpoint{1.765681in}{1.493370in}}{\pgfqpoint{1.770071in}{1.503969in}}{\pgfqpoint{1.770071in}{1.515019in}}%
\pgfpathcurveto{\pgfqpoint{1.770071in}{1.526069in}}{\pgfqpoint{1.765681in}{1.536668in}}{\pgfqpoint{1.757867in}{1.544482in}}%
\pgfpathcurveto{\pgfqpoint{1.750054in}{1.552295in}}{\pgfqpoint{1.739455in}{1.556685in}}{\pgfqpoint{1.728405in}{1.556685in}}%
\pgfpathcurveto{\pgfqpoint{1.717355in}{1.556685in}}{\pgfqpoint{1.706756in}{1.552295in}}{\pgfqpoint{1.698942in}{1.544482in}}%
\pgfpathcurveto{\pgfqpoint{1.691128in}{1.536668in}}{\pgfqpoint{1.686738in}{1.526069in}}{\pgfqpoint{1.686738in}{1.515019in}}%
\pgfpathcurveto{\pgfqpoint{1.686738in}{1.503969in}}{\pgfqpoint{1.691128in}{1.493370in}}{\pgfqpoint{1.698942in}{1.485556in}}%
\pgfpathcurveto{\pgfqpoint{1.706756in}{1.477742in}}{\pgfqpoint{1.717355in}{1.473352in}}{\pgfqpoint{1.728405in}{1.473352in}}%
\pgfpathclose%
\pgfusepath{stroke,fill}%
\end{pgfscope}%
\begin{pgfscope}%
\pgfpathrectangle{\pgfqpoint{0.750000in}{0.375000in}}{\pgfqpoint{4.650000in}{2.265000in}}%
\pgfusepath{clip}%
\pgfsetbuttcap%
\pgfsetroundjoin%
\definecolor{currentfill}{rgb}{0.839216,0.152941,0.156863}%
\pgfsetfillcolor{currentfill}%
\pgfsetlinewidth{1.003750pt}%
\definecolor{currentstroke}{rgb}{0.839216,0.152941,0.156863}%
\pgfsetstrokecolor{currentstroke}%
\pgfsetdash{}{0pt}%
\pgfpathmoveto{\pgfqpoint{1.257039in}{1.469490in}}%
\pgfpathcurveto{\pgfqpoint{1.268089in}{1.469490in}}{\pgfqpoint{1.278688in}{1.473880in}}{\pgfqpoint{1.286502in}{1.481694in}}%
\pgfpathcurveto{\pgfqpoint{1.294316in}{1.489507in}}{\pgfqpoint{1.298706in}{1.500106in}}{\pgfqpoint{1.298706in}{1.511156in}}%
\pgfpathcurveto{\pgfqpoint{1.298706in}{1.522206in}}{\pgfqpoint{1.294316in}{1.532806in}}{\pgfqpoint{1.286502in}{1.540619in}}%
\pgfpathcurveto{\pgfqpoint{1.278688in}{1.548433in}}{\pgfqpoint{1.268089in}{1.552823in}}{\pgfqpoint{1.257039in}{1.552823in}}%
\pgfpathcurveto{\pgfqpoint{1.245989in}{1.552823in}}{\pgfqpoint{1.235390in}{1.548433in}}{\pgfqpoint{1.227577in}{1.540619in}}%
\pgfpathcurveto{\pgfqpoint{1.219763in}{1.532806in}}{\pgfqpoint{1.215373in}{1.522206in}}{\pgfqpoint{1.215373in}{1.511156in}}%
\pgfpathcurveto{\pgfqpoint{1.215373in}{1.500106in}}{\pgfqpoint{1.219763in}{1.489507in}}{\pgfqpoint{1.227577in}{1.481694in}}%
\pgfpathcurveto{\pgfqpoint{1.235390in}{1.473880in}}{\pgfqpoint{1.245989in}{1.469490in}}{\pgfqpoint{1.257039in}{1.469490in}}%
\pgfpathclose%
\pgfusepath{stroke,fill}%
\end{pgfscope}%
\begin{pgfscope}%
\pgfpathrectangle{\pgfqpoint{0.750000in}{0.375000in}}{\pgfqpoint{4.650000in}{2.265000in}}%
\pgfusepath{clip}%
\pgfsetrectcap%
\pgfsetroundjoin%
\pgfsetlinewidth{0.803000pt}%
\definecolor{currentstroke}{rgb}{0.690196,0.690196,0.690196}%
\pgfsetstrokecolor{currentstroke}%
\pgfsetdash{}{0pt}%
\pgfpathmoveto{\pgfqpoint{1.097581in}{0.375000in}}%
\pgfpathlineto{\pgfqpoint{1.097581in}{2.640000in}}%
\pgfusepath{stroke}%
\end{pgfscope}%
\begin{pgfscope}%
\pgfsetbuttcap%
\pgfsetroundjoin%
\definecolor{currentfill}{rgb}{0.000000,0.000000,0.000000}%
\pgfsetfillcolor{currentfill}%
\pgfsetlinewidth{0.803000pt}%
\definecolor{currentstroke}{rgb}{0.000000,0.000000,0.000000}%
\pgfsetstrokecolor{currentstroke}%
\pgfsetdash{}{0pt}%
\pgfsys@defobject{currentmarker}{\pgfqpoint{0.000000in}{-0.048611in}}{\pgfqpoint{0.000000in}{0.000000in}}{%
\pgfpathmoveto{\pgfqpoint{0.000000in}{0.000000in}}%
\pgfpathlineto{\pgfqpoint{0.000000in}{-0.048611in}}%
\pgfusepath{stroke,fill}%
}%
\begin{pgfscope}%
\pgfsys@transformshift{1.097581in}{0.375000in}%
\pgfsys@useobject{currentmarker}{}%
\end{pgfscope}%
\end{pgfscope}%
\begin{pgfscope}%
\pgftext[x=1.097581in,y=0.277778in,,top]{\sffamily\fontsize{10.000000}{12.000000}\selectfont −6}%
\end{pgfscope}%
\begin{pgfscope}%
\pgfpathrectangle{\pgfqpoint{0.750000in}{0.375000in}}{\pgfqpoint{4.650000in}{2.265000in}}%
\pgfusepath{clip}%
\pgfsetrectcap%
\pgfsetroundjoin%
\pgfsetlinewidth{0.803000pt}%
\definecolor{currentstroke}{rgb}{0.690196,0.690196,0.690196}%
\pgfsetstrokecolor{currentstroke}%
\pgfsetdash{}{0pt}%
\pgfpathmoveto{\pgfqpoint{1.756775in}{0.375000in}}%
\pgfpathlineto{\pgfqpoint{1.756775in}{2.640000in}}%
\pgfusepath{stroke}%
\end{pgfscope}%
\begin{pgfscope}%
\pgfsetbuttcap%
\pgfsetroundjoin%
\definecolor{currentfill}{rgb}{0.000000,0.000000,0.000000}%
\pgfsetfillcolor{currentfill}%
\pgfsetlinewidth{0.803000pt}%
\definecolor{currentstroke}{rgb}{0.000000,0.000000,0.000000}%
\pgfsetstrokecolor{currentstroke}%
\pgfsetdash{}{0pt}%
\pgfsys@defobject{currentmarker}{\pgfqpoint{0.000000in}{-0.048611in}}{\pgfqpoint{0.000000in}{0.000000in}}{%
\pgfpathmoveto{\pgfqpoint{0.000000in}{0.000000in}}%
\pgfpathlineto{\pgfqpoint{0.000000in}{-0.048611in}}%
\pgfusepath{stroke,fill}%
}%
\begin{pgfscope}%
\pgfsys@transformshift{1.756775in}{0.375000in}%
\pgfsys@useobject{currentmarker}{}%
\end{pgfscope}%
\end{pgfscope}%
\begin{pgfscope}%
\pgftext[x=1.756775in,y=0.277778in,,top]{\sffamily\fontsize{10.000000}{12.000000}\selectfont −4}%
\end{pgfscope}%
\begin{pgfscope}%
\pgfpathrectangle{\pgfqpoint{0.750000in}{0.375000in}}{\pgfqpoint{4.650000in}{2.265000in}}%
\pgfusepath{clip}%
\pgfsetrectcap%
\pgfsetroundjoin%
\pgfsetlinewidth{0.803000pt}%
\definecolor{currentstroke}{rgb}{0.690196,0.690196,0.690196}%
\pgfsetstrokecolor{currentstroke}%
\pgfsetdash{}{0pt}%
\pgfpathmoveto{\pgfqpoint{2.415969in}{0.375000in}}%
\pgfpathlineto{\pgfqpoint{2.415969in}{2.640000in}}%
\pgfusepath{stroke}%
\end{pgfscope}%
\begin{pgfscope}%
\pgfsetbuttcap%
\pgfsetroundjoin%
\definecolor{currentfill}{rgb}{0.000000,0.000000,0.000000}%
\pgfsetfillcolor{currentfill}%
\pgfsetlinewidth{0.803000pt}%
\definecolor{currentstroke}{rgb}{0.000000,0.000000,0.000000}%
\pgfsetstrokecolor{currentstroke}%
\pgfsetdash{}{0pt}%
\pgfsys@defobject{currentmarker}{\pgfqpoint{0.000000in}{-0.048611in}}{\pgfqpoint{0.000000in}{0.000000in}}{%
\pgfpathmoveto{\pgfqpoint{0.000000in}{0.000000in}}%
\pgfpathlineto{\pgfqpoint{0.000000in}{-0.048611in}}%
\pgfusepath{stroke,fill}%
}%
\begin{pgfscope}%
\pgfsys@transformshift{2.415969in}{0.375000in}%
\pgfsys@useobject{currentmarker}{}%
\end{pgfscope}%
\end{pgfscope}%
\begin{pgfscope}%
\pgftext[x=2.415969in,y=0.277778in,,top]{\sffamily\fontsize{10.000000}{12.000000}\selectfont −2}%
\end{pgfscope}%
\begin{pgfscope}%
\pgfpathrectangle{\pgfqpoint{0.750000in}{0.375000in}}{\pgfqpoint{4.650000in}{2.265000in}}%
\pgfusepath{clip}%
\pgfsetrectcap%
\pgfsetroundjoin%
\pgfsetlinewidth{0.803000pt}%
\definecolor{currentstroke}{rgb}{0.690196,0.690196,0.690196}%
\pgfsetstrokecolor{currentstroke}%
\pgfsetdash{}{0pt}%
\pgfpathmoveto{\pgfqpoint{3.075163in}{0.375000in}}%
\pgfpathlineto{\pgfqpoint{3.075163in}{2.640000in}}%
\pgfusepath{stroke}%
\end{pgfscope}%
\begin{pgfscope}%
\pgfsetbuttcap%
\pgfsetroundjoin%
\definecolor{currentfill}{rgb}{0.000000,0.000000,0.000000}%
\pgfsetfillcolor{currentfill}%
\pgfsetlinewidth{0.803000pt}%
\definecolor{currentstroke}{rgb}{0.000000,0.000000,0.000000}%
\pgfsetstrokecolor{currentstroke}%
\pgfsetdash{}{0pt}%
\pgfsys@defobject{currentmarker}{\pgfqpoint{0.000000in}{-0.048611in}}{\pgfqpoint{0.000000in}{0.000000in}}{%
\pgfpathmoveto{\pgfqpoint{0.000000in}{0.000000in}}%
\pgfpathlineto{\pgfqpoint{0.000000in}{-0.048611in}}%
\pgfusepath{stroke,fill}%
}%
\begin{pgfscope}%
\pgfsys@transformshift{3.075163in}{0.375000in}%
\pgfsys@useobject{currentmarker}{}%
\end{pgfscope}%
\end{pgfscope}%
\begin{pgfscope}%
\pgftext[x=3.075163in,y=0.277778in,,top]{\sffamily\fontsize{10.000000}{12.000000}\selectfont 0}%
\end{pgfscope}%
\begin{pgfscope}%
\pgfpathrectangle{\pgfqpoint{0.750000in}{0.375000in}}{\pgfqpoint{4.650000in}{2.265000in}}%
\pgfusepath{clip}%
\pgfsetrectcap%
\pgfsetroundjoin%
\pgfsetlinewidth{0.803000pt}%
\definecolor{currentstroke}{rgb}{0.690196,0.690196,0.690196}%
\pgfsetstrokecolor{currentstroke}%
\pgfsetdash{}{0pt}%
\pgfpathmoveto{\pgfqpoint{3.734357in}{0.375000in}}%
\pgfpathlineto{\pgfqpoint{3.734357in}{2.640000in}}%
\pgfusepath{stroke}%
\end{pgfscope}%
\begin{pgfscope}%
\pgfsetbuttcap%
\pgfsetroundjoin%
\definecolor{currentfill}{rgb}{0.000000,0.000000,0.000000}%
\pgfsetfillcolor{currentfill}%
\pgfsetlinewidth{0.803000pt}%
\definecolor{currentstroke}{rgb}{0.000000,0.000000,0.000000}%
\pgfsetstrokecolor{currentstroke}%
\pgfsetdash{}{0pt}%
\pgfsys@defobject{currentmarker}{\pgfqpoint{0.000000in}{-0.048611in}}{\pgfqpoint{0.000000in}{0.000000in}}{%
\pgfpathmoveto{\pgfqpoint{0.000000in}{0.000000in}}%
\pgfpathlineto{\pgfqpoint{0.000000in}{-0.048611in}}%
\pgfusepath{stroke,fill}%
}%
\begin{pgfscope}%
\pgfsys@transformshift{3.734357in}{0.375000in}%
\pgfsys@useobject{currentmarker}{}%
\end{pgfscope}%
\end{pgfscope}%
\begin{pgfscope}%
\pgftext[x=3.734357in,y=0.277778in,,top]{\sffamily\fontsize{10.000000}{12.000000}\selectfont 2}%
\end{pgfscope}%
\begin{pgfscope}%
\pgfpathrectangle{\pgfqpoint{0.750000in}{0.375000in}}{\pgfqpoint{4.650000in}{2.265000in}}%
\pgfusepath{clip}%
\pgfsetrectcap%
\pgfsetroundjoin%
\pgfsetlinewidth{0.803000pt}%
\definecolor{currentstroke}{rgb}{0.690196,0.690196,0.690196}%
\pgfsetstrokecolor{currentstroke}%
\pgfsetdash{}{0pt}%
\pgfpathmoveto{\pgfqpoint{4.393551in}{0.375000in}}%
\pgfpathlineto{\pgfqpoint{4.393551in}{2.640000in}}%
\pgfusepath{stroke}%
\end{pgfscope}%
\begin{pgfscope}%
\pgfsetbuttcap%
\pgfsetroundjoin%
\definecolor{currentfill}{rgb}{0.000000,0.000000,0.000000}%
\pgfsetfillcolor{currentfill}%
\pgfsetlinewidth{0.803000pt}%
\definecolor{currentstroke}{rgb}{0.000000,0.000000,0.000000}%
\pgfsetstrokecolor{currentstroke}%
\pgfsetdash{}{0pt}%
\pgfsys@defobject{currentmarker}{\pgfqpoint{0.000000in}{-0.048611in}}{\pgfqpoint{0.000000in}{0.000000in}}{%
\pgfpathmoveto{\pgfqpoint{0.000000in}{0.000000in}}%
\pgfpathlineto{\pgfqpoint{0.000000in}{-0.048611in}}%
\pgfusepath{stroke,fill}%
}%
\begin{pgfscope}%
\pgfsys@transformshift{4.393551in}{0.375000in}%
\pgfsys@useobject{currentmarker}{}%
\end{pgfscope}%
\end{pgfscope}%
\begin{pgfscope}%
\pgftext[x=4.393551in,y=0.277778in,,top]{\sffamily\fontsize{10.000000}{12.000000}\selectfont 4}%
\end{pgfscope}%
\begin{pgfscope}%
\pgfpathrectangle{\pgfqpoint{0.750000in}{0.375000in}}{\pgfqpoint{4.650000in}{2.265000in}}%
\pgfusepath{clip}%
\pgfsetrectcap%
\pgfsetroundjoin%
\pgfsetlinewidth{0.803000pt}%
\definecolor{currentstroke}{rgb}{0.690196,0.690196,0.690196}%
\pgfsetstrokecolor{currentstroke}%
\pgfsetdash{}{0pt}%
\pgfpathmoveto{\pgfqpoint{5.052745in}{0.375000in}}%
\pgfpathlineto{\pgfqpoint{5.052745in}{2.640000in}}%
\pgfusepath{stroke}%
\end{pgfscope}%
\begin{pgfscope}%
\pgfsetbuttcap%
\pgfsetroundjoin%
\definecolor{currentfill}{rgb}{0.000000,0.000000,0.000000}%
\pgfsetfillcolor{currentfill}%
\pgfsetlinewidth{0.803000pt}%
\definecolor{currentstroke}{rgb}{0.000000,0.000000,0.000000}%
\pgfsetstrokecolor{currentstroke}%
\pgfsetdash{}{0pt}%
\pgfsys@defobject{currentmarker}{\pgfqpoint{0.000000in}{-0.048611in}}{\pgfqpoint{0.000000in}{0.000000in}}{%
\pgfpathmoveto{\pgfqpoint{0.000000in}{0.000000in}}%
\pgfpathlineto{\pgfqpoint{0.000000in}{-0.048611in}}%
\pgfusepath{stroke,fill}%
}%
\begin{pgfscope}%
\pgfsys@transformshift{5.052745in}{0.375000in}%
\pgfsys@useobject{currentmarker}{}%
\end{pgfscope}%
\end{pgfscope}%
\begin{pgfscope}%
\pgftext[x=5.052745in,y=0.277778in,,top]{\sffamily\fontsize{10.000000}{12.000000}\selectfont 6}%
\end{pgfscope}%
\begin{pgfscope}%
\pgfpathrectangle{\pgfqpoint{0.750000in}{0.375000in}}{\pgfqpoint{4.650000in}{2.265000in}}%
\pgfusepath{clip}%
\pgfsetrectcap%
\pgfsetroundjoin%
\pgfsetlinewidth{0.803000pt}%
\definecolor{currentstroke}{rgb}{0.690196,0.690196,0.690196}%
\pgfsetstrokecolor{currentstroke}%
\pgfsetdash{}{0pt}%
\pgfpathmoveto{\pgfqpoint{0.750000in}{0.518503in}}%
\pgfpathlineto{\pgfqpoint{5.400000in}{0.518503in}}%
\pgfusepath{stroke}%
\end{pgfscope}%
\begin{pgfscope}%
\pgfsetbuttcap%
\pgfsetroundjoin%
\definecolor{currentfill}{rgb}{0.000000,0.000000,0.000000}%
\pgfsetfillcolor{currentfill}%
\pgfsetlinewidth{0.803000pt}%
\definecolor{currentstroke}{rgb}{0.000000,0.000000,0.000000}%
\pgfsetstrokecolor{currentstroke}%
\pgfsetdash{}{0pt}%
\pgfsys@defobject{currentmarker}{\pgfqpoint{-0.048611in}{0.000000in}}{\pgfqpoint{0.000000in}{0.000000in}}{%
\pgfpathmoveto{\pgfqpoint{0.000000in}{0.000000in}}%
\pgfpathlineto{\pgfqpoint{-0.048611in}{0.000000in}}%
\pgfusepath{stroke,fill}%
}%
\begin{pgfscope}%
\pgfsys@transformshift{0.750000in}{0.518503in}%
\pgfsys@useobject{currentmarker}{}%
\end{pgfscope}%
\end{pgfscope}%
\begin{pgfscope}%
\pgftext[x=0.448039in,y=0.465741in,left,base]{\sffamily\fontsize{10.000000}{12.000000}\selectfont −3}%
\end{pgfscope}%
\begin{pgfscope}%
\pgfpathrectangle{\pgfqpoint{0.750000in}{0.375000in}}{\pgfqpoint{4.650000in}{2.265000in}}%
\pgfusepath{clip}%
\pgfsetrectcap%
\pgfsetroundjoin%
\pgfsetlinewidth{0.803000pt}%
\definecolor{currentstroke}{rgb}{0.690196,0.690196,0.690196}%
\pgfsetstrokecolor{currentstroke}%
\pgfsetdash{}{0pt}%
\pgfpathmoveto{\pgfqpoint{0.750000in}{0.848100in}}%
\pgfpathlineto{\pgfqpoint{5.400000in}{0.848100in}}%
\pgfusepath{stroke}%
\end{pgfscope}%
\begin{pgfscope}%
\pgfsetbuttcap%
\pgfsetroundjoin%
\definecolor{currentfill}{rgb}{0.000000,0.000000,0.000000}%
\pgfsetfillcolor{currentfill}%
\pgfsetlinewidth{0.803000pt}%
\definecolor{currentstroke}{rgb}{0.000000,0.000000,0.000000}%
\pgfsetstrokecolor{currentstroke}%
\pgfsetdash{}{0pt}%
\pgfsys@defobject{currentmarker}{\pgfqpoint{-0.048611in}{0.000000in}}{\pgfqpoint{0.000000in}{0.000000in}}{%
\pgfpathmoveto{\pgfqpoint{0.000000in}{0.000000in}}%
\pgfpathlineto{\pgfqpoint{-0.048611in}{0.000000in}}%
\pgfusepath{stroke,fill}%
}%
\begin{pgfscope}%
\pgfsys@transformshift{0.750000in}{0.848100in}%
\pgfsys@useobject{currentmarker}{}%
\end{pgfscope}%
\end{pgfscope}%
\begin{pgfscope}%
\pgftext[x=0.448039in,y=0.795338in,left,base]{\sffamily\fontsize{10.000000}{12.000000}\selectfont −2}%
\end{pgfscope}%
\begin{pgfscope}%
\pgfpathrectangle{\pgfqpoint{0.750000in}{0.375000in}}{\pgfqpoint{4.650000in}{2.265000in}}%
\pgfusepath{clip}%
\pgfsetrectcap%
\pgfsetroundjoin%
\pgfsetlinewidth{0.803000pt}%
\definecolor{currentstroke}{rgb}{0.690196,0.690196,0.690196}%
\pgfsetstrokecolor{currentstroke}%
\pgfsetdash{}{0pt}%
\pgfpathmoveto{\pgfqpoint{0.750000in}{1.177697in}}%
\pgfpathlineto{\pgfqpoint{5.400000in}{1.177697in}}%
\pgfusepath{stroke}%
\end{pgfscope}%
\begin{pgfscope}%
\pgfsetbuttcap%
\pgfsetroundjoin%
\definecolor{currentfill}{rgb}{0.000000,0.000000,0.000000}%
\pgfsetfillcolor{currentfill}%
\pgfsetlinewidth{0.803000pt}%
\definecolor{currentstroke}{rgb}{0.000000,0.000000,0.000000}%
\pgfsetstrokecolor{currentstroke}%
\pgfsetdash{}{0pt}%
\pgfsys@defobject{currentmarker}{\pgfqpoint{-0.048611in}{0.000000in}}{\pgfqpoint{0.000000in}{0.000000in}}{%
\pgfpathmoveto{\pgfqpoint{0.000000in}{0.000000in}}%
\pgfpathlineto{\pgfqpoint{-0.048611in}{0.000000in}}%
\pgfusepath{stroke,fill}%
}%
\begin{pgfscope}%
\pgfsys@transformshift{0.750000in}{1.177697in}%
\pgfsys@useobject{currentmarker}{}%
\end{pgfscope}%
\end{pgfscope}%
\begin{pgfscope}%
\pgftext[x=0.448039in,y=1.124935in,left,base]{\sffamily\fontsize{10.000000}{12.000000}\selectfont −1}%
\end{pgfscope}%
\begin{pgfscope}%
\pgfpathrectangle{\pgfqpoint{0.750000in}{0.375000in}}{\pgfqpoint{4.650000in}{2.265000in}}%
\pgfusepath{clip}%
\pgfsetrectcap%
\pgfsetroundjoin%
\pgfsetlinewidth{0.803000pt}%
\definecolor{currentstroke}{rgb}{0.690196,0.690196,0.690196}%
\pgfsetstrokecolor{currentstroke}%
\pgfsetdash{}{0pt}%
\pgfpathmoveto{\pgfqpoint{0.750000in}{1.507294in}}%
\pgfpathlineto{\pgfqpoint{5.400000in}{1.507294in}}%
\pgfusepath{stroke}%
\end{pgfscope}%
\begin{pgfscope}%
\pgfsetbuttcap%
\pgfsetroundjoin%
\definecolor{currentfill}{rgb}{0.000000,0.000000,0.000000}%
\pgfsetfillcolor{currentfill}%
\pgfsetlinewidth{0.803000pt}%
\definecolor{currentstroke}{rgb}{0.000000,0.000000,0.000000}%
\pgfsetstrokecolor{currentstroke}%
\pgfsetdash{}{0pt}%
\pgfsys@defobject{currentmarker}{\pgfqpoint{-0.048611in}{0.000000in}}{\pgfqpoint{0.000000in}{0.000000in}}{%
\pgfpathmoveto{\pgfqpoint{0.000000in}{0.000000in}}%
\pgfpathlineto{\pgfqpoint{-0.048611in}{0.000000in}}%
\pgfusepath{stroke,fill}%
}%
\begin{pgfscope}%
\pgfsys@transformshift{0.750000in}{1.507294in}%
\pgfsys@useobject{currentmarker}{}%
\end{pgfscope}%
\end{pgfscope}%
\begin{pgfscope}%
\pgftext[x=0.564412in,y=1.454532in,left,base]{\sffamily\fontsize{10.000000}{12.000000}\selectfont 0}%
\end{pgfscope}%
\begin{pgfscope}%
\pgfpathrectangle{\pgfqpoint{0.750000in}{0.375000in}}{\pgfqpoint{4.650000in}{2.265000in}}%
\pgfusepath{clip}%
\pgfsetrectcap%
\pgfsetroundjoin%
\pgfsetlinewidth{0.803000pt}%
\definecolor{currentstroke}{rgb}{0.690196,0.690196,0.690196}%
\pgfsetstrokecolor{currentstroke}%
\pgfsetdash{}{0pt}%
\pgfpathmoveto{\pgfqpoint{0.750000in}{1.836891in}}%
\pgfpathlineto{\pgfqpoint{5.400000in}{1.836891in}}%
\pgfusepath{stroke}%
\end{pgfscope}%
\begin{pgfscope}%
\pgfsetbuttcap%
\pgfsetroundjoin%
\definecolor{currentfill}{rgb}{0.000000,0.000000,0.000000}%
\pgfsetfillcolor{currentfill}%
\pgfsetlinewidth{0.803000pt}%
\definecolor{currentstroke}{rgb}{0.000000,0.000000,0.000000}%
\pgfsetstrokecolor{currentstroke}%
\pgfsetdash{}{0pt}%
\pgfsys@defobject{currentmarker}{\pgfqpoint{-0.048611in}{0.000000in}}{\pgfqpoint{0.000000in}{0.000000in}}{%
\pgfpathmoveto{\pgfqpoint{0.000000in}{0.000000in}}%
\pgfpathlineto{\pgfqpoint{-0.048611in}{0.000000in}}%
\pgfusepath{stroke,fill}%
}%
\begin{pgfscope}%
\pgfsys@transformshift{0.750000in}{1.836891in}%
\pgfsys@useobject{currentmarker}{}%
\end{pgfscope}%
\end{pgfscope}%
\begin{pgfscope}%
\pgftext[x=0.564412in,y=1.784129in,left,base]{\sffamily\fontsize{10.000000}{12.000000}\selectfont 1}%
\end{pgfscope}%
\begin{pgfscope}%
\pgfpathrectangle{\pgfqpoint{0.750000in}{0.375000in}}{\pgfqpoint{4.650000in}{2.265000in}}%
\pgfusepath{clip}%
\pgfsetrectcap%
\pgfsetroundjoin%
\pgfsetlinewidth{0.803000pt}%
\definecolor{currentstroke}{rgb}{0.690196,0.690196,0.690196}%
\pgfsetstrokecolor{currentstroke}%
\pgfsetdash{}{0pt}%
\pgfpathmoveto{\pgfqpoint{0.750000in}{2.166488in}}%
\pgfpathlineto{\pgfqpoint{5.400000in}{2.166488in}}%
\pgfusepath{stroke}%
\end{pgfscope}%
\begin{pgfscope}%
\pgfsetbuttcap%
\pgfsetroundjoin%
\definecolor{currentfill}{rgb}{0.000000,0.000000,0.000000}%
\pgfsetfillcolor{currentfill}%
\pgfsetlinewidth{0.803000pt}%
\definecolor{currentstroke}{rgb}{0.000000,0.000000,0.000000}%
\pgfsetstrokecolor{currentstroke}%
\pgfsetdash{}{0pt}%
\pgfsys@defobject{currentmarker}{\pgfqpoint{-0.048611in}{0.000000in}}{\pgfqpoint{0.000000in}{0.000000in}}{%
\pgfpathmoveto{\pgfqpoint{0.000000in}{0.000000in}}%
\pgfpathlineto{\pgfqpoint{-0.048611in}{0.000000in}}%
\pgfusepath{stroke,fill}%
}%
\begin{pgfscope}%
\pgfsys@transformshift{0.750000in}{2.166488in}%
\pgfsys@useobject{currentmarker}{}%
\end{pgfscope}%
\end{pgfscope}%
\begin{pgfscope}%
\pgftext[x=0.564412in,y=2.113726in,left,base]{\sffamily\fontsize{10.000000}{12.000000}\selectfont 2}%
\end{pgfscope}%
\begin{pgfscope}%
\pgfpathrectangle{\pgfqpoint{0.750000in}{0.375000in}}{\pgfqpoint{4.650000in}{2.265000in}}%
\pgfusepath{clip}%
\pgfsetrectcap%
\pgfsetroundjoin%
\pgfsetlinewidth{0.803000pt}%
\definecolor{currentstroke}{rgb}{0.690196,0.690196,0.690196}%
\pgfsetstrokecolor{currentstroke}%
\pgfsetdash{}{0pt}%
\pgfpathmoveto{\pgfqpoint{0.750000in}{2.496085in}}%
\pgfpathlineto{\pgfqpoint{5.400000in}{2.496085in}}%
\pgfusepath{stroke}%
\end{pgfscope}%
\begin{pgfscope}%
\pgfsetbuttcap%
\pgfsetroundjoin%
\definecolor{currentfill}{rgb}{0.000000,0.000000,0.000000}%
\pgfsetfillcolor{currentfill}%
\pgfsetlinewidth{0.803000pt}%
\definecolor{currentstroke}{rgb}{0.000000,0.000000,0.000000}%
\pgfsetstrokecolor{currentstroke}%
\pgfsetdash{}{0pt}%
\pgfsys@defobject{currentmarker}{\pgfqpoint{-0.048611in}{0.000000in}}{\pgfqpoint{0.000000in}{0.000000in}}{%
\pgfpathmoveto{\pgfqpoint{0.000000in}{0.000000in}}%
\pgfpathlineto{\pgfqpoint{-0.048611in}{0.000000in}}%
\pgfusepath{stroke,fill}%
}%
\begin{pgfscope}%
\pgfsys@transformshift{0.750000in}{2.496085in}%
\pgfsys@useobject{currentmarker}{}%
\end{pgfscope}%
\end{pgfscope}%
\begin{pgfscope}%
\pgftext[x=0.564412in,y=2.443323in,left,base]{\sffamily\fontsize{10.000000}{12.000000}\selectfont 3}%
\end{pgfscope}%
\begin{pgfscope}%
\pgfsetrectcap%
\pgfsetmiterjoin%
\pgfsetlinewidth{0.803000pt}%
\definecolor{currentstroke}{rgb}{0.000000,0.000000,0.000000}%
\pgfsetstrokecolor{currentstroke}%
\pgfsetdash{}{0pt}%
\pgfpathmoveto{\pgfqpoint{0.750000in}{0.375000in}}%
\pgfpathlineto{\pgfqpoint{0.750000in}{2.640000in}}%
\pgfusepath{stroke}%
\end{pgfscope}%
\begin{pgfscope}%
\pgfsetrectcap%
\pgfsetmiterjoin%
\pgfsetlinewidth{0.803000pt}%
\definecolor{currentstroke}{rgb}{0.000000,0.000000,0.000000}%
\pgfsetstrokecolor{currentstroke}%
\pgfsetdash{}{0pt}%
\pgfpathmoveto{\pgfqpoint{5.400000in}{0.375000in}}%
\pgfpathlineto{\pgfqpoint{5.400000in}{2.640000in}}%
\pgfusepath{stroke}%
\end{pgfscope}%
\begin{pgfscope}%
\pgfsetrectcap%
\pgfsetmiterjoin%
\pgfsetlinewidth{0.803000pt}%
\definecolor{currentstroke}{rgb}{0.000000,0.000000,0.000000}%
\pgfsetstrokecolor{currentstroke}%
\pgfsetdash{}{0pt}%
\pgfpathmoveto{\pgfqpoint{0.750000in}{0.375000in}}%
\pgfpathlineto{\pgfqpoint{5.400000in}{0.375000in}}%
\pgfusepath{stroke}%
\end{pgfscope}%
\begin{pgfscope}%
\pgfsetrectcap%
\pgfsetmiterjoin%
\pgfsetlinewidth{0.803000pt}%
\definecolor{currentstroke}{rgb}{0.000000,0.000000,0.000000}%
\pgfsetstrokecolor{currentstroke}%
\pgfsetdash{}{0pt}%
\pgfpathmoveto{\pgfqpoint{0.750000in}{2.640000in}}%
\pgfpathlineto{\pgfqpoint{5.400000in}{2.640000in}}%
\pgfusepath{stroke}%
\end{pgfscope}%
\end{pgfpicture}%
\makeatother%
\endgroup%

\end{center}
\end{bsp}
Wir diskutieren vollständig den eindimensionalen Fall:
$$x_{k+1} = bx_k,\qquad k=0,1,2,…, \;b\in ℝ.$$
Es gibt sieben Fälle:
\begin{enumerate}
[label=\arabic*)]
\item $b < -1$ 
und man hat eine orientierungsumkehrende Quelle.
\begin{center}
\tikzset{every picture/.style={line width=0.75pt}} %set default line width to 0.75pt        

\begin{tikzpicture}[x=0.75pt,y=0.75pt,yscale=-1,xscale=1]
%uncomment if require: \path (0,300); %set diagram left start at 0, and has height of 300

%Straight Lines [id:da7278447555193781] 
\draw    (100,110) -- (502.79,110) ;


%Straight Lines [id:da0364565703725821] 
\draw    (299.79,100.57) -- (299.79,119.6) ;


\draw   (264.11,102.68) -- (278.32,116.89)(278.32,102.68) -- (264.11,116.89) ;
\draw   (374.11,102.68) -- (388.32,116.89)(388.32,102.68) -- (374.11,116.89) ;
\draw   (117.11,102.89) -- (131.32,117.11)(131.32,102.89) -- (117.11,117.11) ;

% Text Node
\draw (301,89) node   {$0$};
% Text Node
\draw (270,130) node   {$x_{0}$};
% Text Node
\draw (380,130) node   {$x_{1}$};
% Text Node
\draw (130,130) node   {$x_{2}$};
\end{tikzpicture}
\end{center} 

\item $b\in (-1, 0)$ liefert eine orientierungsumkehrende Senke.
\begin{center}
\tikzset{every picture/.style={line width=0.75pt}} %set default line width to 0.75pt        

\begin{tikzpicture}[x=0.75pt,y=0.75pt,yscale=-1,xscale=1]
%uncomment if require: \path (0,300); %set diagram left start at 0, and has height of 300

%Straight Lines [id:da7278447555193781] 
\draw    (100,110) -- (502.79,110) ;


%Straight Lines [id:da0364565703725821] 
\draw    (299.79,100.57) -- (299.79,119.6) ;


\draw   (264.11,102.68) -- (278.32,116.89)(278.32,102.68) -- (264.11,116.89) ;
\draw   (374.11,102.68) -- (388.32,116.89)(388.32,102.68) -- (374.11,116.89) ;
\draw   (117.11,102.89) -- (131.32,117.11)(131.32,102.89) -- (117.11,117.11) ;

% Text Node
\draw (301,89) node   {$0$};
% Text Node
\draw (270,130) node   {$x_{2}$};
% Text Node
\draw (380,130) node   {$x_{1}$};
% Text Node
\draw (130,130) node   {$x_{0}$};
\end{tikzpicture}
\end{center} 
\item $b\in (0,1)$ liefert eine orientierungserhaltende Senke.
\begin{center}


\tikzset{every picture/.style={line width=0.75pt}} %set default line width to 0.75pt        

\begin{tikzpicture}[x=0.75pt,y=0.75pt,yscale=-1,xscale=1]
%uncomment if require: \path (0,300); %set diagram left start at 0, and has height of 300

%Straight Lines [id:da7278447555193781] 
\draw    (100,110) -- (502.79,110) ;


%Straight Lines [id:da0364565703725821] 
\draw    (299.79,100.57) -- (299.79,119.6) ;


\draw   (457.11,102.89) -- (471.32,117.11)(471.32,102.89) -- (457.11,117.11) ;
\draw   (367.11,102.89) -- (381.32,117.11)(381.32,102.89) -- (367.11,117.11) ;
\draw   (327.11,102.89) -- (341.32,117.11)(341.32,102.89) -- (327.11,117.11) ;

% Text Node
\draw (301,89) node   {$0$};
% Text Node
\draw (470,130) node   {$x_{0}$};
% Text Node
\draw (380,130) node   {$x_{1}$};
% Text Node
\draw (330,130) node   {$x_{2}$};

\end{tikzpicture}
\end{center}
\item $b>1$ liefert eine orientierungserhaltende Quelle.
\begin{center}


\tikzset{every picture/.style={line width=0.75pt}} %set default line width to 0.75pt        

\begin{tikzpicture}[x=0.75pt,y=0.75pt,yscale=-1,xscale=1]
%uncomment if require: \path (0,300); %set diagram left start at 0, and has height of 300

%Straight Lines [id:da7278447555193781] 
\draw    (100,110) -- (502.79,110) ;


%Straight Lines [id:da0364565703725821] 
\draw    (299.79,100.57) -- (299.79,119.6) ;


\draw   (457.11,102.89) -- (471.32,117.11)(471.32,102.89) -- (457.11,117.11) ;
\draw   (367.11,102.89) -- (381.32,117.11)(381.32,102.89) -- (367.11,117.11) ;
\draw   (327.11,102.89) -- (341.32,117.11)(341.32,102.89) -- (327.11,117.11) ;

% Text Node
\draw (301,89) node   {$0$};
% Text Node
\draw (470,130) node   {$x_{2}$};
% Text Node
\draw (380,130) node   {$x_{1}$};
% Text Node
\draw (330,130) node   {$x_{0}$};

\end{tikzpicture}
\end{center}
\item Für $b=-1$ ist die Abbildung orienterungsumkeherend und es gilt $f^2(x) = x$ für alle x.
\begin{center}


\tikzset{every picture/.style={line width=0.75pt}} %set default line width to 0.75pt        

\begin{tikzpicture}[x=0.75pt,y=0.75pt,yscale=-1,xscale=1]
%uncomment if require: \path (0,300); %set diagram left start at 0, and has height of 300

%Straight Lines [id:da7278447555193781] 
\draw    (100,110) -- (502.79,110) ;


%Straight Lines [id:da0364565703725821] 
\draw    (299.79,100.57) -- (299.79,119.6) ;


\draw   (378.68,102.89) -- (392.89,117.11)(392.89,102.89) -- (378.68,117.11) ;
\draw   (208.68,102.89) -- (222.89,117.11)(222.89,102.89) -- (208.68,117.11) ;

% Text Node
\draw (301,89) node   {$0$};
% Text Node
\draw (380,130) node   {$x_{0}$};
% Text Node
\draw (220,130) node   {$x_{1}$};
% Text Node
\draw (400,150) node   {$x_{2} =x_{0}$};


\end{tikzpicture}

\end{center}
\item Für $b=0$ ist $f(x) = 0$.
\item Für $b=1$ ist $f(x) = x$.
\end{enumerate}

Um das Verhalten einer nicht-hyperbolischen Abbildung zu beschreiben, müssen wir verstehen, was dynamisch in $E^0$ passiert.\\
$E^0$ zerfällt in 
$$E_1, E_{-1} \text{ und } E_{λ,\ov λ}\text{ mit }|λ| = 1 \quad (λ \ne \pm 1)$$
Falls die Eigenwerte einfach sind, so ergibt sich folgende Dynamik:
\begin{itemize}
\item $E_1$: alle $x\in E^1$ sind Fixpunkte.
\item $E_{-1}$: alle $x\in E^{-1}$, $x\ne 0$, sind 2-periodisch.
\item $E_{λ,\ov{λ}}$: ist $λ= e^{iφ}$, so operiert $A$ auf dem Raum $\lin{\Re v, \Im v}$ wie eine Rotation um den Winkel $φ$,
denn sei $v = v_1 + i v_2$, dann gilt
$$Av = e^{iφ} v = (\cos φ v_1 - \sin φ v_2 ) + i (\sin φ v_1 + \cos φ v_2).$$
Also 
$$ A \begin{pmatrix}
v_1 \\ v_2 
\end{pmatrix} = \begin{pmatrix}
\cos φ  & - \sin φ \\ \sin φ & \cos φ
\end{pmatrix}\begin{pmatrix}
v_1 \\ v_2
\end{pmatrix}.$$
\end{itemize}

\setcounter{section}{5}
\section{Rotationen des Kreises}\label{section:2.6}
in multiplikativer Schreibweise habe wir
$$S^1 = \{ z\in ℂ: |z |=1\} = \{ e^{i2πφ} : φ\in ℝ \}$$
und in additiver Schreibweise
$$S^1 = ℝ/ℤ.$$
Es bezeichne $ℝ_α$ die Rotation um den Winkel $2πα$. 
In multiplikativer Schreibeweise drückt sich das aus durch:
$$R_α z = z_0z, \quad z_0 = e^{i2πα}.$$
In additiver Schreibweise durch
$$R_α x = x+α \mod 1.$$
Für die Iterierten gilt dann
\setcounter{equation}{9}
\begin{equation}
\label{eqn:2.10}
R^n_α z = R_{nα} z = z_0^n z \;\text{ bzw. }\; R_α^n x = x + nα\mod 1
\end{equation}
\end{document}