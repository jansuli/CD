\documentclass[main.tex]{subfiles}

\begin{document}
\setcounter{chapter}{2}

\begin{bem*}
Es gelten die folgenden Aussagen:
\begin{enumerate}[label=(\roman*)]
\item Ein Floquet-Multiplikator ist 1 (vgl. Übung 2).
\item Es sei $φ^t(x)$ der Fluss und $p = \ov x(0)$. Dann gilt
$$Dφ^T(p) = D_xφ^T(p) = e^{TR}.$$ 
\end{enumerate}
\end{bem*}

\begin{proof}
Für den Fluss gilt
\begin{align}
\dt φ^t(x) &= f\left( φ^t(x) \right) \nonumber \\
\Rightarrow \dt Dφ^t(x) &= Df\left( φ^t (x) \right) Dφ^t(x) \label{eqn:2bem} \tag{$\star$}
\end{align}
Somit ist $Dφ^t(p)$ Fundamentallösung von \eqref{eqn:2.5} mit $Dφ^0(p) = I$, da $φ^0(x) = \mathrm{id}$. 
Wegen der vorgehenden Bemerkung 
\begin{equation}
\label{eqn:2bem2}
X(0)= Z(0) = I\Rightarrow X(T) = e^{TR} \tag{$\star \star$}
\end{equation}
folgt 
$$Dφ^T(p) = e^{TR}.$$
\end{proof}

\begin{bem*}
Mit diesem Ergebnis sin Stabilitätsuntersuchungen für periodische Lösungen (Orbits) über die numerische Lösung der Gleichung \eqref{eqn:2.5} mit der Anfangsbedingung $X(0) = I$ möglich.\\
In anderen Worten: Integration von \eqref{eqn:2.5} für $t\in [0,T]$.
\end{bem*}

\begin{bem*}
Es sei $φ^t(x)$ der Fluss und $p\in γ$ v. Dann gilt $Dφ^T(p) = e^{TR}$.
\end{bem*}

\begin{proof}
Für $p\in γ$ ist $φ^t(p) = \overline x( t+ θ )$. 
Es folgt 
\begin{align*}
\dt Dφ^t(p) &\stackrel{\eqref{eqn:2bem}}= Df \left(\ov x ( t+ θ)\right) \cdot Dφ^t (p).
\end{align*}
Daher besitzt $Dφ^t(p)$ die Form 
$$Dφ^t(p) = Z(t+θ)e^{(t+θ)R},$$
wobei $Z(t) e^{tR}$ Lösung der Variationsgleichung \eqref{eqn:2.5} ist, 
denn $Dφ^{t-θ}(p)$ ist Lösung von \eqref{eqn:2.5}.
Mit $Dφ^0(p) = I$ folgt $Z(θ)^{-1} = e^{θR}$.
Damit gilt 
$$Dφ^T(p) = Z(T+θ) e^{(T+θ)R} = Z(θ) e^{θR}e^{TR} = e^{TR}.$$
\end{proof}

Abschließend betrachten wir eine gewöhnliche Differentialgleichung mit zeitabhängiger rechter Seite, also zunächst nicht autonom:
\begin{equation}\label{eqn:2.6}
\dot x = f(x, t) , \qquad (x,t)\in ℝ^n \times ℝ,
\end{equation}
wobei $f$ periodisch in $t$ sei, d.h.
\begin{align*}
f(x,t) &= f(x, t+T) \qquad ∀(x,t).
\end{align*}
Wir können \eqref{eqn:2.6} autonomisieren und umschreiben in das System 
\begin{equation}
\label{eqn:2.7}
\begin{aligned}
\dot x &= f(x, θ), \\
\dot θ &= 1
,\end{aligned}
\end{equation}
wobei jetzt $(x,θ)\in ℝ^n \times S^1$ ist.
Für die Gleichung \eqref{2.7} existiert ein globaler transversaler Schnitt
$$Σ = \left \{ (x, θ)\in ℝ^n \times S^1: θ= θ_0 \right\},$$
denn wegen $\dot θ = 1$ sind alle Lösungen transversal zu $Σ$ in $θ=θ_0$.
Die Poincaré-Abbildung ist daher
$$P\colon Σ\to Σ,\; P(x) = π\circ φ^t(x, θ_0),$$
wobei $φ^t$ der Fluss von \eqref{eqn:2.7} ist und $π\colon ℝ^n\times S^1\to ℝ^n$ eine Projektion ist.
Beachte, dass die Rückkehrzeit konstant gleich $T$ ist.
Auch hier gilt
\begin{enumerate}
[label=(\alph*)]
\item Ein Fixpunkt $p$ von $P$ entspricht einem periodischen Orbit der Periode $T$.
\item Ein periodischer Punkt $p$ der Ordnung $k$ von $P$, d.h. 
$$P^k(p) = p, P^j(p) \ne p, j\in \{1,…, k-1 \},$$
entspricht einer subharmonischen Lösung der Periode $kT$ von \eqref{eqn:2.7}.
\end{enumerate} 
u
\begin{figure}[h!]
\begin{center}
\begin{subfigure}[b]{0.4\textwidth}
\resizebox{\linewidth}{!}{ 


\tikzset{every picture/.style={line width=0.75pt}} %set default line width to 0.75pt        

\begin{tikzpicture}[x=0.75pt,y=0.75pt,yscale=-1,xscale=1]
%uncomment if require: \path (0,300); %set diagram left start at 0, and has height of 300

%Shape: Parallelogram [id:dp804587278263937] 
\draw   (191.65,90) -- (405.5,90) -- (313.85,148) -- (100,148) -- cycle ;
%Shape: Circle [id:dp23172584105668959] 
\draw  [fill={rgb, 255:red, 0; green, 0; blue, 0 }  ,fill opacity=1 ] (250.18,121.57) .. controls (250.18,120.15) and (251.33,119) .. (252.75,119) .. controls (254.17,119) and (255.32,120.15) .. (255.32,121.57) .. controls (255.32,122.99) and (254.17,124.14) .. (252.75,124.14) .. controls (251.33,124.14) and (250.18,122.99) .. (250.18,121.57) -- cycle ;
%Shape: Circle [id:dp04181977269503179] 
\draw  [fill={rgb, 255:red, 0; green, 0; blue, 0 }  ,fill opacity=1 ] (330.18,109.57) .. controls (330.18,108.15) and (331.33,107) .. (332.75,107) .. controls (334.17,107) and (335.32,108.15) .. (335.32,109.57) .. controls (335.32,110.99) and (334.17,112.14) .. (332.75,112.14) .. controls (331.33,112.14) and (330.18,110.99) .. (330.18,109.57) -- cycle ;
%Shape: Circle [id:dp18545005734266118] 
\draw  [fill={rgb, 255:red, 0; green, 0; blue, 0 }  ,fill opacity=1 ] (173.61,135.57) .. controls (173.61,134.15) and (174.76,133) .. (176.18,133) .. controls (177.6,133) and (178.75,134.15) .. (178.75,135.57) .. controls (178.75,136.99) and (177.6,138.14) .. (176.18,138.14) .. controls (174.76,138.14) and (173.61,136.99) .. (173.61,135.57) -- cycle ;
%Curve Lines [id:da888537198628037] 
\draw [color={rgb, 255:red, 255; green, 0; blue, 0 }  ,draw opacity=1 ]   (252.75,119) .. controls (272.79,46.86) and (297.79,29.86) .. (322.79,22.86) .. controls (347.79,15.86) and (387.79,23.86) .. (411.79,51.86) .. controls (435.43,79.44) and (455.18,133.21) .. (440.48,174) ;
\draw [shift={(439.79,175.86)}, rotate = 291.32] [fill={rgb, 255:red, 255; green, 0; blue, 0 }  ,fill opacity=1 ][line width=0.75]  [draw opacity=0] (8.93,-4.29) -- (0,0) -- (8.93,4.29) -- cycle    ;

%Curve Lines [id:da9990098357810433] 
\draw [color={rgb, 255:red, 208; green, 2; blue, 27 }  ,draw opacity=1 ]   (250.98,149.95) .. controls (257.63,218.51) and (299.23,250.88) .. (342.79,252.86) .. controls (386.79,254.86) and (430.79,206.86) .. (439.79,175.86) ;

\draw [shift={(250.79,147.86)}, rotate = 85.1] [fill={rgb, 255:red, 208; green, 2; blue, 27 }  ,fill opacity=1 ][line width=0.75]  [draw opacity=0] (8.93,-4.29) -- (0,0) -- (8.93,4.29) -- cycle    ;
%Curve Lines [id:da5544857271640182] 
\draw    (336.19,137.78) .. controls (341.69,167.38) and (358.96,232.23) .. (404.79,218.86) .. controls (451.79,205.14) and (423.79,59.57) .. (362.79,31.57) .. controls (301.79,3.57) and (257.79,27.86) .. (239.79,38.57) .. controls (221.79,49.29) and (197.79,72.57) .. (176.18,135.57) ;

\draw [shift={(335.79,135.57)}, rotate = 79.96] [fill={rgb, 255:red, 0; green, 0; blue, 0 }  ][line width=0.75]  [draw opacity=0] (8.93,-4.29) -- (0,0) -- (8.93,4.29) -- cycle    ;
%Straight Lines [id:da8651054746864311] 
\draw  [dash pattern={on 4.5pt off 4.5pt}]  (332.75,112.14) -- (335.79,135.57) ;


%Straight Lines [id:da7390310836561813] 
\draw [color={rgb, 255:red, 255; green, 0; blue, 0 }  ,draw opacity=1 ] [dash pattern={on 4.5pt off 4.5pt}]  (252.75,124.14) -- (250.79,147.86) ;



% Text Node
\draw (102,127) node [scale=1.2]  {$Σ$};
% Text Node
\draw (194,133) node   {$q$};
% Text Node
\draw (267,125) node   {$p$};
% Text Node
\draw (352,106) node   {$P( q)$};


\end{tikzpicture}}
\caption{$p$ Fixpunkt von $P$}
\end{subfigure}
\begin{subfigure}[b]{0.4\textwidth}
\resizebox{\linewidth}{!}{ 


\tikzset{every picture/.style={line width=0.75pt}} %set default line width to 0.75pt        

\begin{tikzpicture}[x=0.75pt,y=0.75pt,yscale=-1,xscale=1]
%uncomment if require: \path (0,300); %set diagram left start at 0, and has height of 300

%Shape: Parallelogram [id:dp804587278263937] 
\draw   (191.65,90) -- (405.5,90) -- (313.85,148) -- (100,148) -- cycle ;
%Shape: Circle [id:dp23172584105668959] 
\draw  [fill={rgb, 255:red, 0; green, 0; blue, 0 }  ,fill opacity=1 ] (250.18,121.57) .. controls (250.18,120.15) and (251.33,119) .. (252.75,119) .. controls (254.17,119) and (255.32,120.15) .. (255.32,121.57) .. controls (255.32,122.99) and (254.17,124.14) .. (252.75,124.14) .. controls (251.33,124.14) and (250.18,122.99) .. (250.18,121.57) -- cycle ;
%Shape: Circle [id:dp18545005734266118] 
\draw  [fill={rgb, 255:red, 0; green, 0; blue, 0 }  ,fill opacity=1 ] (173.61,135.57) .. controls (173.61,134.15) and (174.76,133) .. (176.18,133) .. controls (177.6,133) and (178.75,134.15) .. (178.75,135.57) .. controls (178.75,136.99) and (177.6,138.14) .. (176.18,138.14) .. controls (174.76,138.14) and (173.61,136.99) .. (173.61,135.57) -- cycle ;
%Curve Lines [id:da5544857271640182] 
\draw    (176,151.75) .. controls (180.17,187.72) and (214.53,239.83) .. (299.79,252.57) .. controls (386.79,265.57) and (454.79,207.57) .. (461.79,141.57) .. controls (468.79,75.57) and (446.79,21.57) .. (357.79,19.57) .. controls (268.79,17.57) and (253.79,70.57) .. (252.75,124.14) ;

\draw [shift={(175.79,149.57)}, rotate = 85.24] [fill={rgb, 255:red, 0; green, 0; blue, 0 }  ][line width=0.75]  [draw opacity=0] (8.93,-4.29) -- (0,0) -- (8.93,4.29) -- cycle    ;
%Straight Lines [id:da7390310836561813] 
\draw [color={rgb, 255:red, 0; green, 0; blue, 0 }  ,draw opacity=1 ] [dash pattern={on 0.84pt off 2.51pt}]  (252.75,124.14) -- (250.79,147.86) ;


%Curve Lines [id:da7617831953152889] 
\draw    (251.01,150.03) .. controls (255.39,185.85) and (294.81,233.91) .. (344.79,201.57) .. controls (395.79,168.57) and (370.21,30.43) .. (281.21,28.43) .. controls (192.21,26.43) and (177.21,79.43) .. (176.18,133) ;

\draw [shift={(250.79,147.86)}, rotate = 85.24] [fill={rgb, 255:red, 0; green, 0; blue, 0 }  ][line width=0.75]  [draw opacity=0] (8.93,-4.29) -- (0,0) -- (8.93,4.29) -- cycle    ;
%Straight Lines [id:da22874792354178752] 
\draw [color={rgb, 255:red, 0; green, 0; blue, 0 }  ,draw opacity=1 ] [dash pattern={on 0.84pt off 2.51pt}]  (176.18,133) -- (175.79,149.57) ;



% Text Node
\draw (102,127) node [scale=1.2]  {$Σ$};
% Text Node
\draw (267,125) node   {$p_{0}$};
% Text Node
\draw (196,134) node   {$p_{1}$};


\end{tikzpicture}
}
\caption{$P(p_0) = p_1$ und $P(p_1) = p_0$.}
\end{subfigure}
\end{center}
\end{figure}

\setcounter{section}{4}
\section{Lineare Abbildungen}
\label{section:2.5}
Wir untersuchen das dynamische Verhalten von Systemen
\begin{equation}
\label{eqn:2.8}
x_{k+1} = Ax_k, \quad k=0,1,2,…,
\end{equation}
wobei $A \in ℝ^{n\times n}$. 
Zunächst benötigen wir dazu ein technisches Hilfsmittel:

\setcounter{satz}{17}
\begin{satz}\label{2.18}
Es sei $ρ(A)$ der Spektralradius der Matrix $A$. Dann existiert zu jedem $δ>0$ eine Norm $\| • \|_δ$, so dass 
$$\|A\|_δ < ρ(A) + δ.$$
\end{satz}
\begin{bem}\label{2.19}
Da im $ℝ^n$ alle Normen äquivalent sind, gibt für jede Norm $\|•\|$ und jedes $ε>0$ ein $C_ε$, so dass für alle $v\in ℝ^n$ gilt
\begin{equation}
\label{2.19}
\| A^n v \| \le C_ε \left( ρ(A) + ε \right)^n \| v \|,
\end{equation}
denn 
$$\| A^n v\| \le C_ε\| A^n \|_ε \| v \| \le C_ε \left( ρ(A) + ε \right)^n \| v\|.$$
\end{bem}

\begin{korollar}\label{2.20}
Falls sämtliche Eigenwerte der linearen Abbildung $A\colon ℝ^n\to ℝ^n$ vom Betrage her kleiner als 1 sind, so konvergieren die (positiven) Iterierten 
$$y_k = A^k x \text{ zu $x\in ℝ^n$}$$
exponentiell gegen den Ursprung 0. 
D.h. es existieren ein $c>0$ und ein $λ\in [0,1)$, so dass für $k\ge 0$ gilt:
$$\|y_k \|< cλ^k.$$
\end{korollar}

\begin{bsp}\label{2.21}
$ $\\[-1em]
\begin{enumerate}
[label=(\alph*)]
\item Es sei 
$$A = \begin{pmatrix}
λ & 0\\ 0& λ
\end{pmatrix} \text{ mit $λ\in (0,1)$.}$$
% skizze x-y-Achsen mit diskreten iterierten in qurtiel 1 & 4 die gegen 0 gehen exponentiell (kleinere löcken)
\item es sei 
$$A = \begin{pmatrix}
λ & 0
\\0 & μ
\end{pmatrix}\text{ mit $λ,μ\in (0,1), μ< λ$.}$$
Korollar \ref{2.20} impliziert, dass sämtliche Anfangswerte unter Iteration gegen die $0$ konvergieren. Man erhält folgendes Bild:

\begin{center}
%% Creator: Matplotlib, PGF backend
%%
%% To include the figure in your LaTeX document, write
%%   \input{<filename>.pgf}
%%
%% Make sure the required packages are loaded in your preamble
%%   \usepackage{pgf}
%%
%% Figures using additional raster images can only be included by \input if
%% they are in the same directory as the main LaTeX file. For loading figures
%% from other directories you can use the `import` package
%%   \usepackage{import}
%% and then include the figures with
%%   \import{<path to file>}{<filename>.pgf}
%%
%% Matplotlib used the following preamble
%%   \usepackage{fontspec}
%%   \setmainfont{DejaVu Serif}
%%   \setsansfont{DejaVu Sans}
%%   \setmonofont{DejaVu Sans Mono}
%%
\begingroup%
\makeatletter%
\begin{pgfpicture}%
\pgfpathrectangle{\pgfpointorigin}{\pgfqpoint{6.000000in}{4.000000in}}%
\pgfusepath{use as bounding box, clip}%
\begin{pgfscope}%
\pgfsetbuttcap%
\pgfsetmiterjoin%
\definecolor{currentfill}{rgb}{1.000000,1.000000,1.000000}%
\pgfsetfillcolor{currentfill}%
\pgfsetlinewidth{0.000000pt}%
\definecolor{currentstroke}{rgb}{1.000000,1.000000,1.000000}%
\pgfsetstrokecolor{currentstroke}%
\pgfsetdash{}{0pt}%
\pgfpathmoveto{\pgfqpoint{0.000000in}{0.000000in}}%
\pgfpathlineto{\pgfqpoint{6.000000in}{0.000000in}}%
\pgfpathlineto{\pgfqpoint{6.000000in}{4.000000in}}%
\pgfpathlineto{\pgfqpoint{0.000000in}{4.000000in}}%
\pgfpathclose%
\pgfusepath{fill}%
\end{pgfscope}%
\end{pgfpicture}%
\makeatother%
\endgroup%

\end{center}
% skizze / plot x-y Achse (ggf. mit Python plotten
% caption: Solche Phasen-Bilder nennt man auch Knoten nach Englisch node
\end{enumerate}
\end{bsp}
\end{document}